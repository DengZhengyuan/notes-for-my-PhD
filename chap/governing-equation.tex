\chapter{Governing Equation of Fluid Dynamics}
\citet{anderson2008computational}.
\section{Substantial Derivative}
In cartesian space, the unit vectors along the $x$, $y$, and $z$ axes are $\bm{i}$, $\bm{j}$, and $\bm{k}$ in this chapter. 
The vector velocity field in this cartesian space is given by 
\begin{equation}
    \bm{u}(x,y,z,t) = u\bm{i} + v\bm{j} + w\bm{k}. 
\end{equation}
The scalar density field is given by 
\begin{equation}
    \rho = \rho (x,y,z,t). 
\end{equation}

At time $t_1$, the infinitesimal fluid element is located at point 1. At a later time $t_2$, the same element has moved to point 2. Then, expand the density function in a Taylor series about point 1, 
\begin{equation}
\begin{aligned}
    \rho_2 = 
    \rho_1 
    + \left(\frac{\partial \rho}{\partial x}\right)_1 (x_2 -x_1)
    &+ \left(\frac{\partial \rho}{\partial y}\right)_1 (y_2 -y_1)
    + \left(\frac{\partial \rho}{\partial z}\right)_1 (z_2 -z_1) \\
    &+ \left(\frac{\partial \rho}{\partial t}\right)_1 (t_2 -t_1)
    + (\text{high order terms}).
\end{aligned}
\end{equation}
\begin{equation}
\begin{aligned}
    \frac{{D} \rho}{{D} t} &\equiv
    \lim _{t_{2} \to t_{1}} 
    \frac{\rho_{2}-\rho_{1}}{t_{2}-t_{1}} \\
    &= \frac{\partial \rho}{\partial t}
    +u \frac{\partial \rho}{\partial x}
    +v \frac{\partial \rho}{\partial y}
    +w \frac{\partial \rho}{\partial z} \\
    &= \frac{\partial \rho}{\partial t}
    +\bm{u} \cdot \grad \rho
\end{aligned}
\end{equation}

\begin{definition}
    Substantial derivative. 
    \begin{equation}
        \frac{{D} \phi}{{D} t} 
        \equiv 
        \frac{\partial \phi}{\partial t}
        +\bm{u} \cdot \grad \phi
    \end{equation}
    where $\phi$ is a scalar field. 
    $D/Dt$ called substantial derivative. 
    Physically, it tracks the rate of change of a fluid element with respect to time. 
    The first term called local derivative, which shows the rate of change at a fixed point; 
    the second term called convective derivative, which shows the rate of change as the fluid element moving from a point to another point due to the non-uniformity of the flow distribution. 
    \qed 
\end{definition}

\section{Physical Meaning of Divergence of Velocity}
There is a constant mass fluid element with variables of density. 
So, its volume and surface will change with time. 
Consider an infinitesimal element of the surface $dS$ moving at the local velocity $\bm{u}$. 
The change in the volume of the control volume, 
\begin{equation}
\begin{aligned}
    \Delta V &= (\bm{u}\,\Delta t)\cdot \bm{n}dS \\
    &= (\bm{u}\,\Delta t)\cdot \bm{dS}.
\end{aligned}
\end{equation}
where $\bm{n}dS = \bm{dS}$ is the project area. 
Then, the total change of volume of the whole control volume is 
\begin{equation}
    \oiint_S (\bm{u}\,\Delta t)\cdot \bm{dS}. 
\end{equation}

If this integral is divided by $\Delta t$, the result is physically the time rate of change of the control volume, which is 
\begin{equation}
    \frac{D V}{D t} =
    \oiint_S \bm{u}\cdot \bm{dS}. 
\end{equation}
Applying the divergence theorem to it, 
\begin{equation}
    \frac{D V}{D t} =
    \iiint_V \div \bm{u} \, dV. 
\end{equation}
Assume the control volume is shrunk to a very small volume, $\delta V$, 
\begin{equation}
\begin{aligned}
    \frac{D(\delta V)}{Dt} 
    &= \div \bm{u} \, \delta V \\
    \div \bm{u} 
    &= \frac{1}{\delta V}\, \frac{D(\delta V)}{Dt}.
\end{aligned}
\end{equation}
Accordingly, the physical meaning of $(\div \bm{u})$ is shown in the right side, which is the time rate of change of the volume of a moving fluid element, per unit volume. 

\section{The Continuity Equation}
Fundamental physical principle: mass is conserved. 
\subsection{The model of a moving fluid element}
The mass of the fluid element is fixed, and is given by $\delta m$. 
Then, 
\begin{equation}
    \delta m = \rho \, \delta V.
\end{equation}
Because the mass is fixed, which means it does not change with time, 
\begin{align}
    \frac{D(\delta m)}{Dt} &= 0, \\
    \Longrightarrow \quad 
    \frac{D(\rho\,\delta V)}{Dt} 
    &= \delta V\,\frac{D\rho}{Dt} 
    + \rho\, \frac{D(\delta V)}{Dt} = 0, \\
    \Longrightarrow \quad 
    \frac{D\rho}{Dt}
    &+ \rho \left[
        \frac{1}{\delta V}\, \frac{D(\delta V)}{Dt}
        \right] \\
    \Longrightarrow \quad 
    \frac{D\rho}{Dt}
    &+ \rho \,(\div \bm{u}) = 0.
    \label{eq: continuity equation in non-conservation form}
\end{align}
Eq.~\eqref{eq: continuity equation in non-conservation form} is the continuity equation in non-conservation form. 

\subsection{The model of a finite control volume fixed in space}
The mass in the control volume is conserved, 
\begin{equation}
\begin{aligned}
    \left(\substack{
        \text{Net mass flow out of control} \\
        \text{volume through surface }S}\right) 
        &= 
    \left(\substack{
        \text{Time rate of decrease of} \\
        \text{mass inside control volume}}\right), \\
    B &= C.
\end{aligned}
\end{equation}
Then, 
\begin{align}
    B &= \oiint_S \rho \, \bm{u} \cdot \bm{dS}, \\
    C &= -\frac{\partial}{\partial t} 
    \iiint_V \rho\, dV. 
\end{align}
Combine the above equations, 
\begin{equation}
    \frac{\partial}{\partial t} 
    \iiint_V \rho\, dV 
    + \oiint_S \rho \, \bm{u} \cdot \bm{dS} 
    = 0.
\end{equation}
This is the integral form of the continuity equation, as well as in conservation form.

If we apply the divergence theorem to the second term, 
\begin{equation*}
    \iiint_V \left[
        \frac{\partial \rho}{\partial t}
        + \div (\rho\,\bm{u})
        \right]\,dV = 0
\end{equation*}
Then, the continuity equation on conservation form is 
\begin{equation}
    \frac{\partial \rho}{\partial t}
    + \div (\rho\,\bm{u})
    = 0. 
    \label{eq: continuity equation on conservation form}
\end{equation}

\subsection{Transformation between two models}
\begin{equation}
    \div (\rho \, \bm{u}) 
    \equiv 
    \rho \, \div \bm{u}
    + \bm{u} \cdot \grad \rho . 
\end{equation}
Then, the Eq.~\eqref{eq: continuity equation on conservation form} can be rewritten as 
\begin{equation}
    \frac{\partial \rho}{\partial t} 
    + \rho \, \div \bm{u}
    + \bm{u} \cdot \grad \rho
    = \frac{D\rho}{Dt}
    + \rho \,(\div \bm{u})
    = 0, 
\end{equation}
which have same form with Eq.~\eqref{eq: continuity equation in non-conservation form}. 

\section{The Momentum Equation}
Physical principle: Newton's second law, 
\begin{equation}
    \bm{F} = m\, \bm{a}.
    \label{eq: newton's second law}
\end{equation}

There are two sources of the force term, $\bm{F}$: 
\begin{enumerate}
    \item Body force, which act directly on the surface of the fluid element, such as gravitational, electric, and magnetic forces. 
    \item Surface force, which act directly on the surface of the fluid element: a) pressure; b) shear and normal stress.
\end{enumerate}

\paragraph{Body force. }
Let us consider the $x$-component of the forces: 
\begin{equation}
    (\text{body force}) = \rho \, f_x\, (dx\, dy\, dz).
\end{equation}

\paragraph{Surface force. }
xxxxxxxxx

\begin{figure}[h!]
    \raggedleft
    \includegraphics{fig/x-direc_forces_in_infinitesimal_fluid_element}
    \caption{Infinitesimal fluid element with forces in $x$-direction}
    \label{fig: infinitesimal fluid element with forces in x-direction}
\end{figure}

With the above in mind, for the moving fluid element, 
\begin{equation}
\begin{aligned}
    \left(\substack{\text{Net surface force} \\
                    \text{in }x\text{-direction}} \right)
    &= \left[
        p- \left( 
            p+\frac{\partial p}{\partial x}\, dx 
            \right)
        \right] \, dy\,dz \\
    &+ \left[
            \left( 
                \tau_{xx}
                +\frac{\partial \tau_{xx}}{\partial x}\, dx 
            \right)
            - \tau_{xx}
        \right] \, dy\,dz \\
    &+ \left[
            \left( 
                \tau_{yx}
                +\frac{\partial \tau_{yx}}{\partial y}\, dy 
            \right)
            - \tau_{yx}
        \right] \, dx\,dz \\
    &+ \left[
            \left( 
                \tau_{zx}
                +\frac{\partial \tau_{zx}}{\partial z}\, dz 
            \right)
            - \tau_{zx}
        \right] \, dx\,dy. 
\end{aligned}
\end{equation}
Then, the total force in the $x$-direction is 
\begin{equation}
    F_x = \left(
        -\frac{\partial p}{\partial x} 
        +\frac{\partial \tau_{xx}}{\partial x}
        +\frac{\partial \tau_{yx}}{\partial y}
        +\frac{\partial \tau_{zx}}{\partial z}
    \right) \, dx\,dy\,dz
    + \rho\, f_x\, dx\,dy\,dz. 
\end{equation}

For the right-hand side of Eq.~\eqref{eq: newton's second law}, 
\begin{equation}
\begin{aligned}
    m &= \rho \, dx\,dy\,dz, \\
    a_x &= \frac{Du}{Dt}. 
\end{aligned}
\end{equation}

Accordingly, we obtain the $x$-component of momentum equation for a viscous flow, which is Eq.~\eqref{eq: ns x non-conservation}. In a similar fashion, the $y$ and $z$ components can be obtained as Eq.~\eqref{eq: ns y non-conservation} and \eqref{eq: ns z non-conservation}.
\begin{subequations}
\begin{align}
    \rho \, \frac{Du}{Dt}
    = -\frac{\partial p}{\partial x}
    +\frac{\partial \tau_{xx}}{\partial x}
    +\frac{\partial \tau_{yx}}{\partial y}
    +\frac{\partial \tau_{zx}}{\partial z}
    + \rho\, f_x. 
    \label{eq: ns x non-conservation} \\
    \rho \, \frac{Dv}{Dt}
    = -\frac{\partial p}{\partial y}
    +\frac{\partial \tau_{xy}}{\partial x}
    +\frac{\partial \tau_{yy}}{\partial y}
    +\frac{\partial \tau_{zy}}{\partial z}
    + \rho\, f_y, 
    \label{eq: ns y non-conservation} \\
    \rho \, \frac{Dw}{Dt}
    = -\frac{\partial p}{\partial z}
    +\frac{\partial \tau_{xz}}{\partial x}
    +\frac{\partial \tau_{yz}}{\partial y}
    +\frac{\partial \tau_{zz}}{\partial z}
    + \rho\, f_z. 
    \label{eq: ns z non-conservation} 
\end{align}
\label{eq: ns non-conservation} 
\end{subequations}

The three equations are partial differential equations obtained directly from an application of the fundamental physical principle to an infinitesimal fluid element. 
Moreover, since this fluid element is moving with flow, the equations are in non-conservation form. 
They are scalar equations, and are called the Navier-Stokes equations. 

The Navier-Stokes equations can be obtained in conservation form as follows. 
\begin{equation}
\begin{aligned}
    \frac{\partial (\rho \, u)}{\partial t} 
    &= \rho\, \frac{\partial u}{\partial t}
    + u\, \frac{\partial \rho}{\partial t}, \\
    \rho\, \frac{\partial u}{\partial t} 
    &= \frac{\partial (\rho \, u)}{\partial t} 
    - u\, \frac{\partial \rho}{\partial t}. 
    \label{eq: ns trans 1}
\end{aligned}
\end{equation}

\begin{equation}
\begin{aligned}
    \grad\cdot (\rho\, u \, \bm{u}) 
    &= \rho \, \bm{u} \cdot \grad u
    + u\,\grad\cdot (\rho\,\bm{u}) \\
    \rho \, \bm{u} \cdot \grad u
    &= \grad\cdot (\rho\,u\, \bm{u}) 
    - u\,\grad\cdot (\rho\,\bm{u}). 
    \label{eq: ns trans 2}
\end{aligned}
\end{equation}

Substitute the Eq.~\eqref{eq: ns trans 1} and \eqref{eq: ns trans 2} into the left-hand side of Eq.~\eqref{eq: ns x non-conservation}, 
\begin{equation}
\begin{aligned}
    \rho\,\frac{Du}{Dt} 
    &= \pdv{(\rho\,u)}{t}
    - u\, \pdv{\rho}{t} 
    + \div (\rho\,u \,\bm{u})
    - u\, \div (\rho\, \bm{u}), \\
    &= \pdv{(\rho\,u)}{t}
    + \div (\rho\,u \,\bm{u}) 
    -u\,
    \underbrace{\cancelto{0}{\left[
        \div (\rho\, \bm{u}) + \pdv{\rho}{t}
    \right]}.}_\text{continuity equation}
\end{aligned}
\end{equation}
Then, the Navier-Stokes equations in conservation form are 
\begin{subequations}
\begin{align}
    \pdv{(\rho\,u)}{t}
    + \div (\rho\,u \,\bm{u}) 
    = -\frac{\partial p}{\partial x}
    +\frac{\partial \tau_{xx}}{\partial x}
    +\frac{\partial \tau_{yx}}{\partial y}
    +\frac{\partial \tau_{zx}}{\partial z}
    + \rho\, f_x, 
    \label{eq: ns x conservation} \\
    \pdv{(\rho\,v)}{t}
    + \div (\rho\,v \,\bm{u})
    = -\frac{\partial p}{\partial y}
    +\frac{\partial \tau_{xy}}{\partial x}
    +\frac{\partial \tau_{yy}}{\partial y}
    +\frac{\partial \tau_{zy}}{\partial z}
    + \rho\, f_y, 
    \label{eq: ns y conservation} \\
    \pdv{(\rho\,w)}{t}
    + \div (\rho\,w \,\bm{u})
    = -\frac{\partial p}{\partial z}
    +\frac{\partial \tau_{xz}}{\partial x}
    +\frac{\partial \tau_{yz}}{\partial y}
    +\frac{\partial \tau_{zz}}{\partial z}
    + \rho\, f_z. 
    \label{eq: ns z conservation} 
\end{align}
\label{eq: ns conservation}
\end{subequations}

\subsection{Some definitions and notations}
\subsubsection{Kronecker product and outer product}
\paragraph{For matrixs. }
In tensor, the Kronecker product, denoted by $\otimes$, is an operation on two matrices of arbitrary size resulting in a block matrix \citep{wiki:Kronecker}. 
It is a generalization of the outer product (which is denoted by the same symbol) from vectors to matrices, and gives the matrix of the tensor product with respect to a standard choice of basis. 
Kronecker product is different from cross product. 

\begin{definition}
    Kronecker product

    If $\bm{A}$ is an $(m_1 \times m_2)$ matrix and $\bm{B}$ is a $(n_1 \times n_2)$ matrix, the Kronecker product $\bm{A} \otimes \bm{B}$ is the $(m_1\,n_1 \times m_2\, n_2)$ block matrix: 
    \begin{equation}
        \bm{A} \otimes \bm{B} = a_{ij}b_{kl} = 
        \begin{pmatrix}
            {a_{11} \bm{B}} & {a_{12} \bm{B}} & {\cdots} & {a_{1 m_{2}} \bm{B}} \\ 
            {a_{21} \bm{B}} & {a_{22} \bm{B}} & {\cdots} & {a_{2 m_{2}} \bm{B}} \\ 
            {\vdots} & {\vdots} & {\ddots} & {\vdots} \\ 
            {a_{m_1 1} \bm{B}} & {a_{m_1 2} \bm{B}} & {\cdots} & {a_{m_1 m_2} \bm{B}}
        \end{pmatrix}.
    \end{equation}
    \qed
\end{definition}

\paragraph{For vectors. }
In linear algebra, the outer product of two coordinate vectors is a matrix. If the two vectors have dimensions $n$ and $m$, then their outer product is an $n \times m$ matrix \citet{wiki:outer_product}. 
\begin{definition}
    Outer product

    Give two vectors, 
    \begin{align}
    \bm{u} &= \left(u_{1}, u_{2}, \cdots, u_{m}\right), \\ 
    \bm{v} &= \left(v_{1}, v_{2}, \cdots, v_{n}\right). 
    \end{align}
    Their outer product $\bm{u} \otimes \bm{v}$ is defined as the $n \times m$ matrix $\bm{A}$ obtained by multiplying each element of $\bm{u}$ by each element of $\bm{v}$: 
    \begin{equation}
    \bm{u} \otimes \bm{v} =
    u_i\,v_j =
    \begin{pmatrix}
        {u_{1} v_{1}} & {u_{1} v_{2}} & {\dots} & {u_{1} v_{n}} \\ 
        {u_{2} v_{1}} & {u_{2} v_{2}} & {\dots} & {u_{2} v_{n}} \\ 
        {\vdots} & {\vdots} & {\ddots} & {\vdots} \\ 
        {u_{m} v_{1}} & {u_{m} v_{2}} & {\dots} & {u_{m} v_{n}}
    \end{pmatrix}
    \end{equation}
    \qed
\end{definition}
Therefore, in the second term of Eq.~\eqref{eq: ns conservation}, 
\begin{equation}
\begin{aligned}
    \rho\,\bm{u} \otimes \bm{u} = \rho\,
    \begin{pmatrix}
        uu & vu & wu \\
        uv & vv & wv \\
        uw & vw & ww 
    \end{pmatrix}.
\end{aligned}
\end{equation}

There are several equivalent terms and notations for outer product: 
\begin{enumerate}
    \item the dyalic product of two vector ia denoted by $\bm{a\,b}$, 
    \item the tensor product of two vectors is denoted by $\bm{a} \otimes \bm{b}$. 
\end{enumerate}

\subsubsection{Cauchy stress tensor}
In continuum mechanics, the Cauchy stress tensor $\bm{\sigma}$, true stress tensor, or simply called the stress tensor is a second order tensor named after Augustin-Louis Cauchy \citep{wiki:stresstensor}. 
The tensor consists of nine components $\sigma _{ij}$ that completely define the state of stress at a point inside a material in the deformed state, placement, or configuration.
\begin{equation}
    \bm{\sigma} = 
    \begin{pmatrix}
        {\sigma_{x x}} & {\sigma_{x y}} & {\sigma_{x z}} \\
        {\sigma_{y x}} & {\sigma_{y y}} & {\sigma_{y z}} \\ 
        {\sigma_{z x}} & {\sigma_{z y}} & {\sigma_{z z}}
    \end{pmatrix}
    = 
    \begin{pmatrix}
        {p+ \tau_{xx}} & {\tau_{x y}} & {\tau_{x z}} \\ 
        {\tau_{y x}} & {p+ \tau_{yy}} & {\tau_{y z}} \\ 
        {\tau_{z x}} & {\tau_{z y}} & {p+ \tau_{zz}}
    \end{pmatrix}, 
\end{equation}
where $\sigma_i$ is the normal stress, and $\tau_{ij}$ is the shear stress. 

Then, we can rewrite the Eq.~\eqref{eq: ns non-conservation} and \eqref{eq: ns conservation} as 
\begin{align}
    \rho \, \frac{D \bm{u}}{D t}
    &= -\grad p + \div \bm{\tau} + \rho \bm{g}, 
    \\
    \pdv{(\rho \, \bm{u})}{t} + \div (\rho \, \bm{u} \otimes \bm{u})
    &= -\grad p + \div \bm{\tau} + \rho \bm{g}. 
\end{align}