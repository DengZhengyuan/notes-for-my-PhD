\chapter{Vector Calculus}
This chapter records the notes for vector calculus from a book by \citet{matthews1998vector} and \citet{moukalled2016finite}.

\section{Vector Operations}
\subsection{Dot product}

\subsection{Cross product (vector product)}

\subsection{Scalar triple product}

\subsection{Gradient}

\subsection{Divergence}

\section{Tensor Operations}
\subsection{General tensor}

\subsection{Dyadic product}

\subsection{Tensor dot vector}

\subsection{divergence of tensor}

\subsection{double dot product}

\section{Theorems}
\subsection{Gradient Theorem for Line Integral}

\subsection{Green's theorem}

\subsection{Stokes' theorem}

\subsection{Divergence theorem}




% -------------------------------------------------------------------
% OOOOOOOOOOOOOOOOOOOOOOOOOLD
% -------------------------------------------------------------------

\section{Vector Algebra}
\subsection{Dot product}
The dot product or scalar product of two vectors is a scalar quantity.
\begin{equation}
    \bm{a} \cdot \bm{b} = |\bm{a}| |\bm{b}| \cos \theta
\end{equation}
\begin{itemize}
    \item $ \bm{a}\cdot \bm{b} = \bm{b}\cdot \bm{a} $.
    \item The quantity $|\bm{b}|\cos \theta$ represents the component of the vector $\bm{b}$ in the direction of the vector $\bm{a}$.
    \item $ \bm{a}\cdot \bm{b} = a_1 b_1 + a_2 b_2 + a_3 b_3 $.
    \item $\bm{e}_1 \cdot \bm{e}_2 = 0$. 
\end{itemize}
    
\subsection{Cross product}
The cross product or vector product of two vectors is a vector quantity. 
The magnitude is $|\bm{a}| |\bm{b}| \sin \theta$, and the direction is perpendicular to them in a right-handed sense. 
Then,
\begin{equation}
    \bm{a} \times \bm{b} 
    = |\bm{a}| |\bm{b}| \sin \theta\,\bm{u}
    =   
        \begin{vmatrix}
            \bm{e}_1 & \bm{e}_2 & \bm{e}_3 \\
            \bm{a}_1 & \bm{a}_2 & \bm{a}_3 \\
            \bm{b}_1 & \bm{b}_2 & \bm{b}_3 
        \end{vmatrix}
\end{equation}
where $\bm{u}$ is a unit vector perpendicular to them in a right-handed sense. 
\begin{itemize}
    \item $\bm{a} \times \bm{b} = - \bm{b} \times \bm{a}$.
    \item $\bm{e}_1 \times \bm{e}_2 = 1$. 
\end{itemize}

\subsection{Scalar triple product}
The scalar triple product is defined to be 
\begin{equation}
    \bm{a} \cdot \bm{b} \times \bm{c}
    =
    \bm{a} \cdot (\bm{b} \times \bm{c})
    =
        \begin{vmatrix}
            \bm{a}_1 & \bm{a}_2 & \bm{a}_3 \\
            \bm{b}_1 & \bm{b}_2 & \bm{b}_3 \\
            \bm{c}_1 & \bm{c}_2 & \bm{c}_3 
        \end{vmatrix}
    ,
\end{equation}
written $[\bm{a},\;\bm{b},\;\bm{c}]$. The brackets are unnecessary.

\begin{itemize}
    \item If any two of the vectors are equal, the scalar triple product is zero.
    \item $ \bm{a} \cdot \bm{b} \times \bm{c}
            = \bm{a} \times \bm{b} \cdot \bm{c} $.
    \item $ \bm{a} \cdot \bm{b} \times \bm{c}
            = \bm{b} \cdot \bm{c} \times \bm{a} 
            = \bm{c} \cdot \bm{a} \times \bm{b} $.
\end{itemize}

\subsection{Vector triple product}
The vector triple product is 
$\bm{a} \times (\bm{b}\times \bm{c})$. 
The brackets are important. 
It can expanded as
\begin{equation}
    \bm{a} \times (\bm{b}\times \bm{c})
    = (\bm{a} \cdot \bm{c}) \bm{b}
    - (\bm{a} \cdot \bm{b}) \bm{c}.
\end{equation}
More information see \citealp[pg.~16]{matthews1998vector}.

\section{Line, Surface, and Volume Integrals}
\subsection{Line integrals}
\begin{example}
    A particle moves along a curve path $C$ by force $\bm{F}(\bm{r})$. 
    $\bm{r}$ is the position vector, $\bm{r} = (x,\;y,\;z)$. 
    What is the total amount of work?
    \begin{equation}
        \lim_{N\to \infty} \sum_{i=1}^N \bm{F}_i \cdot \bm{dr}_i
        = \int_C \bm{F} \cdot \bm{dr}.
    \end{equation}
    \qed
\end{example}
\paragraph{Evaluation.}
Line integrals are evaluated by using a parameter, time ($t$), together with a formula giving the value of the position vector $\bm{r}$ in terms of $t$.
\begin{equation}
    \int_C \bm{F} \cdot \bm{dr}
    = \int \bm{F} \cdot \frac{\bm{dr}}{dt} dt 
\end{equation}
where 
\begin{equation}
    \frac{\bm{dr}}{dt} 
    = \left( \frac{dx}{dt},\; \frac{dy}{dt},\; \frac{dz}{dt} \right).
\end{equation}

Line integrals sometimes occur over curves that are close. 
In this case, the integral is written using the symbol $\oint$. 

\paragraph{Conservative vector fields. }
\begin{definition}
    Conservative vector fields. 
    
    A vector field $\bm{F}$ is said to be conservative if it has the property that the line integral of $\bm{F}$ around any closed curve $C$ is zero:
    \begin{equation}
        \oint_C \bm{F} \cdot \bm{dr} = 0.
    \end{equation}
    \qed
\end{definition}
The line integral of $\bm{F}$ along a curve only depends on the endpoints of the curve, not on the path taken by the curve.

\paragraph{Other forms of line integrals. }
\begin{equation}
    \int_C \phi \bm{dr}
    \quad \mathrm{and} \quad
    \int_C \bm{F} \times \bm{dr}. 
\end{equation}

\subsection{Surface integrals}
\begin{example}
    Suppose that fluid flows with velocity $\bm{u}(\bm{r},t)$ through a pipe with $A$ cross-sectional area. 
    What is the total volume of fluid passing through the pipe per unit time?
    
    Suppose the first case: the velocity is directed parallel to the walls of the pipe, with speed $|\bm{u}| = U_0$. 
    In this case, the fluid moves along the pipe as if it were a solid block. 
    In a time $t$, the fluid moves a distance $U_0 t$, so the volume is $U_0 t A$. 
    Thus, the flow rate, or flux is 
    \begin{equation}
        Q = \frac{U_0 t A}{t} = U_0 A.  
    \end{equation}

    Suppose the second case: the velocity is again directed parallel to the walls, but the speed depends on the position within the pipe: 
    $|\bm{u}| = U_0 (x,y)$.
    Besides, the pipe has a square cross-section. 
    Then, the surface element $dS = dx \, dy$. 
    Thus, 
    \begin{align}
        dQ &= U_0(x,y) \, dS = U_0(x,y) \, dx\,dy, \\
        \Longrightarrow \quad
        Q &= \iint_S U_0(x,y) \, dx\,dy. 
    \end{align}

    In the case 1 and 2, the fluid flow direction is perpendicular to the surface. 
    The third case where the vector field $\bm{u}$ and the surface $S$ are both arbitrary. 
    Therefore, only the component of $\bm{u}$ perpendicular to $dS$ contributes to the flux across $dS$.
    Here, it is necessary to introduce a normal vector $\bm{n}$ to the surface $dS$.
    \textbf{The component of $\bm{u}$ perpendicular to $dS$ is then the component of $\bm{u}$ in the direction of $\bm{n}$, which is just $\bm{u} \cdot \bm{n}$.} 
    Therefore, 
    \begin{align}
        dQ &= \bm{u} \cdot \bm{n} \, dS, \\
        \Longrightarrow \quad
        Q &= \iint_S \bm{u} \cdot \bm{n} \, dS. 
    \end{align}

    If the surface is closed, the integral could be written as 
    \begin{equation}
        \oiint_S \bm{u} \cdot \bm{n} \, dS.
    \end{equation}

    \qed 
\end{example}

\paragraph{Evaluation. }
For the second case, if  the square surface given by 
$0 \leq x \leq 1, \; 0 \leq y \leq 1$.
\begin{equation}
    Q = \iint_S U_0(x,y) \, dx\,dy 
      = \int^1_0 \int^1_0 U_0(x,y) \, dx\,dy.
\end{equation}

For the third case, the surface $S$ is curved and can be written in terms of two parameters, $v$ and $w$, so the position vector $\bm{r} = \bm{r}(v,w)$. 
To evaluate the surface integral we need an expression for $\bm{n} \, dS$. 
The cross product of two vectors gives a vector perpendicular to both an with a magnitude equal to the area of the parallelogram create by two vectors. 
Thus, 
\begin{equation}
    \iint_S \bm{u} \cdot \bm{n} \, dS
    = \iint_S \bm{u} \cdot 
    \frac{\partial \bm{r}}{\partial v} \times \frac{\partial \bm{r}}{\partial w} \, dv \, dw.
\end{equation}

\subsection{Volume integrals}
\begin{example}
    Suppose that an object of volume $V$ has a density $\rho$, and the density is a function of position, $\rho = \rho (\bm{r})$.
    What is the total mass ($M$) of the object?
    \begin{equation}
        \iiint_V \rho dV 
        = \lim_{N\to \infty} \sum^N_{i=1} \rho(\bm{r}_i)\,\delta V_i.
    \end{equation}

    Volume integrals can also be used to compute the volumes of objects, in which case $\rho =1$.
    \begin{equation}
        \iiint_V \bm{u} \, dV
    \end{equation}
    where $\bm{u}$ is a vector field.
\end{example}

\section{Gradient, Divergence, and Curl}
\subsection{Taylor series in more than one variable}
For single variable,
\begin{equation}
\begin{aligned}
    f(x) 
    &= f(a)+(x-a) \frac{d f}{d x}(a)+\frac{(x-a)^{2}}{2 !} \frac{d^{2} f}{d x^{2}}(a)+\cdots \\
    &= \sum^\infty_{n=0} \frac{(x-a)^n}{n!} \, \frac{d^n f}{dx^n}\,(a).
\end{aligned}
\end{equation}
For $f(x,y)$, 
\begin{equation}
    \delta f
    =\delta x \frac{\partial f}{\partial x}
    +\delta y \frac{\partial f}{\partial y}
    +\frac{(\delta x)^{2}}{2 !} \frac{\partial^{2} f}{\partial x^{2}}
    +\frac{(\delta y)^{2}}{2 !} \frac{\partial^{2} f}{\partial y^{2}}
    +\delta x \delta y \frac{\partial^{2} f}{\partial x \partial y}
    +\cdots
\end{equation}

\subsection{Gradient of a scalar field}
The gradient of a scalar field $f$ is a vector field, with a direction that is perpendicular to the level surface, pointing in the direction of increasing $f$, with a magnitude equal to the rate of change of $f$ in the direction. 
\begin{equation}
    \bnabla f=\frac{\partial f}{\partial x} \bm{e}_{1}+\frac{\partial f}{\partial y} \bm{e}_{2}+\frac{\partial f}{\partial z} \bm{e}_{3}
\end{equation}

To find the rate of change of $f$ in the direction of the unit vector $\bm{u}$, set $\bm{dr} = \bm{u} \,ds$ where $ds$ is the distance along $\bm{u}$. 
Then, 
\begin{equation}
    \frac{d f}{d s}=\bnabla f \cdot \bm{u}.
\end{equation}
This called directional derivative of $f$.

The symbol $\bnabla$ can be interpreted as a vector differential operator, 
\begin{equation}
    \bnabla=\left(\frac{\partial}{\partial x}, \frac{\partial}{\partial y}, \frac{\partial}{\partial z}\right).
\end{equation}

\subsubsection{Conservative fields and potentials}
There is a very important link between the gradient of a scalar field and a conservative vector field.

\begin{theorem}
    Suppose that a vector field $\bm{F}$ is related to a scalar field $\phi$ by $ \bm{F}=\bnabla \phi $ exists everywhere in some region $D$. 
    Then $\bm{F}$ is conservative within $D$. 
    Conversely, if $\bm{F}$ is conservative, then $\bm{F}$ can be written as the gradient of a scalar field, 
    $ \bm{F}=\bnabla \phi $.
    \qed
\end{theorem}

If a vector field $\bm{F}$ is conservative, the corresponding scalar field $\phi$ which obeys $ \bm{F}=\bnabla \phi $ is called the potential for $\bm{F}$.

\subsection{Divergence of a vector field}
The divergence of a vector field $\bm{u}$ is a scalar field. 
Its value at a point $P$ is defined by 
\begin{equation}
    \mathrm{div} \,\bm{u}=\lim _{\delta V \to 0} 
    \frac{1}{\delta V} \oiint_{\delta S} \bm{u} \cdot \bm{n} \,d S
\end{equation}
where $\delta V$ is a small volume enclosing $P$ with surface $\delta S$ and $\bm{n}$ is the outward pointing normal to $\delta S$. 

In the Cartesian coordinator, 
\begin{equation}
    \begin{aligned}
        \mathrm{div} \bm{u}
        =\frac{\partial u_{1}}{\partial x}+\frac{\partial u_{2}}{\partial y}+\frac{\partial u_{3}}{\partial z}
        = \left(\frac{\partial}{\partial x}, \frac{\partial}{\partial y}, \frac{\partial}{\partial z}\right) \cdot\left(u_{1}, u_{2}, u_{3}\right)
        = \bnabla \cdot \bm{u}.
    \end{aligned}
\end{equation}

\paragraph{Physical interpretation. }
Physically, this corresponds to the amount of flux of the vector field $\bm{u}$ out of $\delta V$ divided by the volume $\delta V$. 
The physical definition of the divergence gives an intuitive meaning in terms of the flux of the vector field out of a small closed surface. 
This can also be interpreted as the rate of `expansion' or `stretching' of the vector field. 

\paragraph{Laplacian of a scalar field. }
Suppose that a scalar field $\phi$ is twice differentiable. 
\begin{equation}
    \bnabla \cdot \bnabla \phi = \nabla^2 \phi
\end{equation}
is a new scalar field. 

The formula for $\nabla^2 \phi$ can be found by combining the formulae for div and grad: 
\begin{equation}
    \begin{aligned}
    \nabla^{2} \phi 
    &=\frac{\partial}{\partial x}\left(\frac{\partial \phi}{\partial x}\right)
    +\frac{\partial}{\partial y}\left(\frac{\partial \phi}{\partial y}\right)
    +\frac{\partial}{\partial z}\left(\frac{\partial \phi}{\partial z}\right) \\ 
    &=\frac{\partial^{2} \phi}{\partial x^{2}}+\frac{\partial^{2} \phi}{\partial y^{2}}+\frac{\partial^{2} \phi}{\partial z^{2}}. 
    \end{aligned}
\end{equation}
This formula means the Laplacian of $\phi$ is just the sum of the second partial derivative of $\phi$. 
The equation $\nabla^2 \phi = 0$ is known as Laplace's equation. 

\subsection{Curl of a vector field}
The curl of a vector field $\bm{u}$ is a vector field. Its component in the direction of the unit vector $\bm{n}$ 
\begin{equation}
    \bm{n} \cdot \mathrm{curl} \, \bm{u}
    =\lim _{\delta S \to 0} 
    \frac{1}{\delta S} 
    \oint_{\delta C} \bm{u} \cdot \bm{d} \bm{r}, 
\end{equation}
where $\delta S$ is a small surface element perpendicular to $\bm{n}$, $\delta C$ is the closed curve forming the boundary of $\delta S$ and $\delta C$ and $\bm{n}$ are oriented in a right-handed sense. 
\begin{equation}
\begin{aligned}
    \mathrm{curl} \, \bm{u}
    &=\left(\frac{\partial u_{3}}{\partial y}-\frac{\partial u_{2}}{\partial z}, 
    \frac{\partial u_{1}}{\partial z}-\frac{\partial u_{3}}{\partial x}, 
    \frac{\partial u_{2}}{\partial x}-\frac{\partial u_{1}}{\partial y}\right) \\
    &=  \begin{vmatrix}
        {\bm{e}_{1}} & {\bm{e}_{2}} & {\bm{e}_{3}} \\ 
        {\sfrac{\partial}{\partial x}} & {\sfrac{\partial}{\partial y}} & {\sfrac{\partial}{\partial z}} \\ 
        {u_{1}} & {u_{2}} & {u_{3}}
        \end{vmatrix} \\
    &= \bnabla \times \bm{u}
\end{aligned}
\end{equation}

\paragraph{Physical interpretation. }
$\bnabla \times \bm{u}$ is related to the rotation or twisting of the vector field $\bm{u}$. 
\begin{example}
    For vector field $\bm{w} = (0,x,0)$ as the velocity of a fluid, $\bnabla \times \bm{w}= (0,0,1)$. 
    Then a small particle placed in this fluid will rotate in an anticlockwise sense as it moves with the fluid, since at any point the velocity component in the y direction to the right of the particle is greater than that on the left. 
    This rotation is about an axis in the z direction. 
    Thus, the rate of rotation depends on the magnitude of $\bnabla \times \bm{w}$, and the axis of rotation is in the direction of $\bnabla \times \bm{w}$.
    \qed
\end{example}

\paragraph{Curl and vonservative vector fields. }
Any vector field that can be written as the gradient of a scalar field is irrotational:
\begin{equation}
    \bnabla \times \bnabla \phi=\mathbf{0}.  
\end{equation}

\section{Integral Theorems}
\subsection{Divergence theorem}
\begin{theorem}
    Let $\bm{u}$ be a continuously differentiable vector field, defined in a volume $V$. 
    Let $S$ be the closed surface forming the boundary of $V$ and let $\bm{n}$ be the unit outward normal to $S$. 
    The the divergence theorem states that 
    \begin{equation}
        \iiint_{V} \bnabla \cdot \bm{u} \, d V
        =\oiint_{S} \bm{u} \cdot \bm{n} \, d S .
    \end{equation}
    \qed
\end{theorem}

The divergence theorem is referred to as Gauss's theorem.
Roughly speaking, the divergence theorem states that the total amount of expansion of $\bm{u}$ within the volume $V$ is equal to the flux of $\bm{u}$ out of the surface $S$. 

There is a derivation of the law for conservation of mass of a fluid shows on the \cite[pg. 85]{matthews1998vector}. 

\subsection{Stokes's theorem}
Let $C$ be a closed curve which forms the boundary of a surface $S$. 
Then for a continuously differentiable vector field $\bm{u}$, Stokes's theorem states that
\begin{equation}
    \iint_{S} \bnabla \times \bm{u} \cdot \bm{n} \, d S
    =\oint_{C} \bm{u} \cdot \bm{dr},
\end{equation}
where the direction of the line integral around $C$ and the normal $\bm{n}$ are oriented in a right-handed sense. 


\section{Cartesian Tensors}
