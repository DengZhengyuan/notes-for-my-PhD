\chapter{Analytical Reaction Models in Fluidized Bed}
The models are based on CSTR, PFR with internal and external mass transfer. 

The ozone decomposition reaction can be considered as isothermal pseudo-first-order irreversible reaction: 
\begin{align}
    (-\mathcal{R}) = -\frac{dC}{dt} = k\,C
\end{align}
where $(-\mathcal{R})$ is chemical reaction (ozone decomposition) rate, 
and $k \; [\si{1/m}]$ is reaction constant. 

\section{Original CSTR and PFR models}
\begin{figure}[h!]
    \raggedleft
    \includegraphics{fig/CSTR+PFR}
    \caption{Schematics of CSTR and PFR: the CSTR is on the left, and the right one is PFR; the blue lines represent the gas}
    \label{fig: CSTR + PFR} 
\end{figure}

\paragraph{CSTR. }
Balance in mole flow:
$[\si{mol/s}]$
\begin{equation*}
    \text{enters} - \text{leaves}
    = \text{accumulations} + \text{reaction}
\end{equation*}
\begin{equation}
    U_g C_i A - U_g C_o A 
    = V_t \, \cancel{\frac{dC}{dt}} 
    + V_t \,(\mathcal{-R})
    \label{eq: kinetics - original CSTR}
\end{equation}
where
\begin{itemize}
    \item $U_g$ $[\si{m/s}]$ is the superficial gas velocity;
    \item $C$ $[\si{mol/m^3}]$ is the reactant concentration, and $i$, $o$ represent inlet and outlet, respectively;
    \item $A$ $[\si{m^2}]$ is the cross-sectional area of the reactor;
    \item $V_t$ $[\si{m^3}]$ is the total volume of the reactor.
\end{itemize}

Under the steady state, the solution for the Eq.~\eqref{eq: kinetics - original CSTR} is 
\begin{equation}
    \frac{C_o}{C_i} = 1-X_\text{CSTR} 
    = \frac{1}{1+\left(\dfrac{k\,H}{U_g}\right)}
    \label{eq: solution - original - CSTR}
\end{equation}
where $k$ is reaction constant based on the volume of the reactor; 
$H$ is the length of the reactor.

\paragraph{PFR. }
Balance in mole flow:
$[\si{mol/s}]$
\begin{equation*}
    \text{enters} - \text{leaves}
    = \text{accumulations} + \text{reaction}
\end{equation*}
\begin{align}
    U_g A C|_z - U_g A C|_{z+\Delta z}
     & = \Delta V \, \cancel{\frac{\partial C}{\partial t}} 
       + \Delta V \,(\mathcal{-R})\\
    \Longrightarrow \quad
    U_g \, \frac{\partial C}{\partial z}
     & = k\,C.
    \label{eq: kinetics - original PFR}
\end{align}

Under the steady state, the solution for the Eq.~\eqref{eq: kinetics - original PFR} is 
\begin{equation}
    \frac{C_o}{C_i} = 1-X_\text{PFR} 
    = \exp{\left( -\frac{k\, H}{U_g} \right)}
    \label{eq: solution - original - PFR}
\end{equation}

\section{Homogeneous Models for Fixed Bed}
In the fixed bed, there are catalyst particles in the reactor.  
Also, the assumptions for the homogeneous models are 
\begin{enumerate}
    \item no internal and external mass transfer;
    \item the concentration of reactants and temperature at the surface of the particle is the same as the bulk gas phase.
\end{enumerate}
Under this circumstance, $k_v\; [\si{m^3/(s\cdot m^3\;cat.)}]$ is defined as the reaction constant based on the catalyst volume. 
Then, 
\begin{equation}
    V \,(\mathcal{-R}) 
    = \underbrace{V \, \varepsilon_s}_{V_s} k_v\,C
\end{equation}
where $\varepsilon_s$ is the solid holdups, 
and $V_s$ is the volume of the solid phase. 
Therefore, the solutions for the two models in Eq.~\eqref{eq: solution - original - CSTR} and \eqref{eq: solution - original - PFR} can be written as 
\begin{align}
    \text{CSTR: }\quad
    \frac{C_o}{C_i} &= 1-X_\text{CSTR} 
    = \frac{1}
    {1+\left(\dfrac{k_v\, \varepsilon_s \, H}{U_g}\right)},
    \\
    \text{PFR: }\quad
    \frac{C_o}{C_i} &= 1-X_\text{PFR} 
    = \exp{\left( -\frac{k_v\, \varepsilon_s \, H}{U_g} \right)}.
\end{align}

The real length that gas passed is shorter than $H$ as the solid phase exists. The solutions are rewritten as 
\begin{align}
    \text{CSTR: }\quad
    \frac{C_o}{C_i} &= 1-X_\text{CSTR} 
    = \frac{1}{1+\left(\dfrac{k_v\, \varepsilon_s \,(1-\varepsilon_s) \, H}{U_g}\right)},
    \\
    \text{PFR: }\quad
    \frac{C_o}{C_i} &= 1-X_\text{PFR} 
    = \exp{\left( -\frac{k_v\, \varepsilon_s \, (1-\varepsilon_s) \, H}{U_g} \right)}.
\end{align}

\section{Heterogeneous Models}
Be different from the homogeneous models, the internal and external diffusion is considered. 
There are some assumptions: 
\begin{enumerate}
    \item $U_g$ is constant.
    \item Solid particles are spherical with mean diameter $d_p$.
\end{enumerate}

\subsection{Gas phase}
In the gas phase models, the external mass transfer should be considered. 
From the Fick's law: 
\begin{equation}
    \bm{J} = -D_m \, \nabla \bm{C}
\end{equation}
where $\bm{J}\; [\si{mol/(m^2\cdot s)}]$ is the diffusion flux: the amount of substance per unit area per unit time; 
$D_m \; [\si{m/s}]$ is the diffusivity.
\begin{align}
    \text{external mass transfer term} =
    \underbrace{D_m\,\frac{(C_g-C_s)}{\delta}}_{\text{Fick's law}}
    \,
    A_s 
    \,
    \frac{V_s}{V_s}\,\frac{V_t}{V_t}
    \qquad  \left[\frac{\si{mol}}{\cancel{\si{m^2}} \cdot \si{s}}\right]
            \,
            \cancel{\si{m^2}}
    \label{eq: external dif. - fixed bed}
\end{align}
where $\delta$ is the film thickness, 
$A_s$ is the total surface area of the catalyst.
\begin{definition}
    $k_s,\; a_s$.
    \begin{enumerate}
        \item $k_s\; [\si{m^3/(m^2\cdot s)}] = D_m/\delta$ is the external mass transfer coefficient.
        \item $a_s\; [\si{1/m}] = A_s/V_s$ is the surface area per unit volume of particle, for sphere is $6/d_p$.
    \end{enumerate}
    \qed
\end{definition}
Consequently, 
\begin{align}
    \text{R.H.S.} =
    k_s \, a_s \, \varepsilon_s (C_g - C_s)
    \label{eq: kinetics - accumulation term}
\end{align}

Combination of Eq.~\eqref{eq: kinetics - original CSTR}, \eqref{eq: kinetics - original PFR}, and Eq.~\eqref{eq: kinetics - accumulation term}: 
\begin{align}
    \text{CSTR: }
    & U_g\, C_{g,i} - U_g\, C_{g,o} 
    = k_s \, a_s \, \varepsilon_s \,(1-\varepsilon_s) (C_g - C_s) H 
    \label{eq: CSTR gas phase} \\ 
    \text{PFR: }
    & U_g\,\frac{\partial C}{\partial z}
    = k_s \, a_s \, \varepsilon_s \,(1-\varepsilon_s) (C_g - C_s) 
    \label{eq: PFR gas phase}
\end{align}

\subsection{Solid phase}
\begin{align*}
    \text{external diffusion}
    = 
    \text{reaction with internal diffusion}
    \qquad \left[\si{\frac{mol}{m^3\cdot s}}\right]
\end{align*}
\begin{align}
    D_m\, \frac{(C_g - C_s)}{\delta}\, A_s
    &= \eta \, k_v \, C_s \, V_s \\
    \Longrightarrow \quad 
    k_s \, a_s (C_g - C_s) 
    &= k_v \, C_s \, \eta 
    \label{eq: solid phase}
\end{align}
where $k_v\, [\si{1/s}]$ is the reaction constant based on the volume of catalyst particle; $\eta\; [-]$ is the internal mass transfer resistance.

\subsection{Rearrangement}
Rearrangement of Eq.~\eqref{eq: solid phase}:
\begin{equation}
    C_s = \frac{k_s\, a_s}{k_s\, a_s + k_v\,\eta}\, C_g
    \label{eq: rearranged solid phase}
\end{equation}

Combination of Eq.~\eqref{eq: CSTR gas phase} and \eqref{eq: PFR gas phase}, and Eq.~\eqref{eq: rearranged solid phase}:
\begin{align}
    \text{CSTR:} \quad
    & \frac{C_{g,o}}{C_{g,i}} = 1- X_\text{CSTR}
    = \frac{1}{1+ \left(
        \dfrac{k_v\,\eta\,k_s\,a_s\,\varepsilon_s\,(1-\varepsilon_s)\,H}{U_g(k_s\, a_s + k_v\,\eta)}
        \right)} 
    \label{eq: CSTR in fixed bed}\\
    \text{PFR:} \quad
    & U_g\,\frac{dC_{g}}{dz} + \frac{k_s\,a_s\,k_v\,\eta}{k_s\,a_s + k_v\,\eta}\,\varepsilon_s\,(1-\varepsilon_s)\,C_g = 0 \\
    \Longrightarrow \quad 
    & \frac{C_{g,o}}{C_{g,i}} = 1- X_\text{PFR}
    = \exp \left[
        - \dfrac{k_v\,\eta\,k_s\,a_s\,\varepsilon_s\,(1-\varepsilon_s)\,H}{U_g(k_s\, a_s + k_v\,\eta)}  
    \right]
    \label{eq: PFR in fixed bed}
\end{align}

\subsection{Calculation}
In the models above, the external diffusion coefficient and the effectiveness factor due to internal diffusion need to be obtained.

For external diffusion coefficient \citep[pg. 691]{fogler2016element}, $k_s$, 
\begin{align}
    Sh &= \frac{k_s\, d_p}{D_m} 
        = 2.0 + 0.63 Re^{\sfrac{1}{2}}Sc^{\sfrac{1}{3}},
    \\
    Re_p &= \frac{\rho_g\, d_p\, U}{\mu_g},
    \\
    Sc &= \frac{\mu_g}{\rho_g\,D_m}
\end{align}
where 
\begin{itemize}
    \item $Sh$ is Sherwood number. 
    \item $Re$ is Reynolds number. 
    \item $Sc$ is Schmidt number. 
    \item $\rho_g \; [\si{kg/m^3}]$ is gas density. 
    \item $\mu_g \; \si{[kg/(m\cdot s)]}$ is gas viscosity. 
    \item $U\;[\si{m/s}]$ is free stream velocity or relative velocity between particles and the gas. 
    If the solids circulation rates are fairly low, it equals to particle terminal velocity \citep{jiang1991baffle}. 
    \item $D_m\;[\si{m^2/s}]$ is molecular diffusion coefficient of ozone.  
\end{itemize}

As for effectiveness factor, $\eta$, 
\begin{align}
    \eta &= \frac{3}{\phi^2}(\phi \coth{\phi}-1),
    \label{eq: eta for first order reaction} \\
    \eta &= \frac{1}{\phi}\left(
        \frac{1}{\tanh{3\phi}} - \frac{1}{3 \phi}    
    \right)
    \label{eq: eta for irreversible first order reaction}
\end{align}
and
\begin{equation*}
    \phi = \frac{d_p}{6}\sqrt{\frac{k_v}{D_e}}
\end{equation*}
where $D_e\,[\si{m^2/s}]$ is effective diffusion coefficient of ozone inside the particle \citep[pg. 721]{fogler2016element}. 
For a first order reaction with spherical catalyst particle. 
The effectiveness factor can be calculated by Eq.~\eqref{eq: eta for irreversible first order reaction} that reported by \citet{li2017reaction} and \citet{jiang1991baffle}, or by Eq.~\eqref{eq: eta for first order reaction} that reported by \citet{fogler2016element}.

\section{Models for Fluidized Bed and Contact Efficiency Factor}
Comparing with the fixed bed, the surface area of the catalyst is lower in the fluidized bed due to the aggregation phenomenon \citep{wang2015performance}. 
Thus, a new term called contact efficiency factor is introduced. 

\begin{definition}
    $\alpha\; [-]$ is the contact efficiency factor, which is a fraction of the external surface area of catalysts available for the diffused ozone reactant from the gas phase:
    \begin{align*}
        \frac{\text{total particle surface area}}{\text{particle surface area available to contact with ozone}}
    \end{align*}
    \qed
\end{definition}
For the heterogeneous models, Eq.~\eqref{eq: external dif. - fixed bed} can be written as 
\begin{align*}
    \text{external mass transfer term} =
    D_m\,\frac{(C_g-C_s)}{\delta}
    \quad \times
    \underbrace{\alpha\, A_s}_{\substack{\text{available surface}\\ \text{area of particles}}} 
    \times \quad
    \frac{V_s}{V_s}\,\frac{V_t}{V_t}
\end{align*}
Thus, 
\begin{align}
    \text{CSTR:} \quad
    & \frac{C_{g,o}}{C_{g,i}} = 1- X_\text{CSTR}
    = \frac{1}{1+ \alpha\,\left(
        \dfrac{k_v\,\eta\,k_s\,a_s\,\varepsilon_s\,(1-\varepsilon_s)\,H}{U_g(k_s\, a_s + k_v\,\eta)}
        \right)},
    \label{eq: CSTR in CFB}\\
    \text{PFR:} \quad
    & U_g\,\frac{dC_{g}}{dz} + \frac{k_s\,a_s\,k_v\,\eta}{k_s\,a_s + k_v\,\eta}\,\varepsilon_s\,(1-\varepsilon_s)\,C_g = 0, \\
    \Longrightarrow \quad 
    & \frac{C_{g,o}}{C_{g,i}} = 1- X_\text{PFR}
    = \exp \left[
        - \alpha\,\dfrac{k_v\,\eta\,k_s\,a_s\,\varepsilon_s\,(1-\varepsilon_s)\,H}{U_g(k_s\, a_s + k_v\,\eta)}  
    \right].
    \label{eq: PFR in CFB}
\end{align}

Slightly changing the definition for the $\alpha$, it can be used in the homogeneous models,
\begin{align}
    \text{CSTR: }\quad
    \frac{C_o}{C_i} &= 1-X_\text{CSTR} 
    = \frac{1}{1+\alpha\,\left(\dfrac{k_v\, \varepsilon_s \,(1-\varepsilon_s) \, H}{U_g}\right)},
    \label{eq: homo - CSTR}
    \\
    \text{PFR: }\quad
    \frac{C_o}{C_i} &= 1-X_\text{PFR} 
    = \exp{\left( -\alpha\,\frac{k_v\, \varepsilon_s \, (1-\varepsilon_s) \, H}{U_g} \right)}.
    \label{eq: homo - PFR}
\end{align}


\section{Comparation and Damköhler number}
\begin{definition}
    Damköhler number

    \citet{jiang1991baffle, liujs2016}: 
    \begin{equation}
        Da = \frac{k_v\,\varepsilon_s\, H}{U_g}, 
    \end{equation}

    \citet{li2011catalytic, wang2015performance}: 
    \begin{equation}
        Da = \frac{k_v\,\varepsilon_s\,(1-\varepsilon_s) \, H}{U_g}
    \end{equation}
    where $k_v$ is the reaction constant based on volume of catalyst particles; 
    in the study of Li, Wang, and Liu, $k_v$ is also called apparent reaction constant.
    \qed
\end{definition}

Accordingly, Eq.~\eqref{eq: CSTR in CFB} and \eqref{eq: PFR in CFB} can be written as
\begin{align}
    1- X_\text{CSTR}
    &= \frac{1}{1+\alpha\,Da\,\left(
        \dfrac{k_s\,a_s\,\eta}{k_s\,a_s + k_v\,\eta}
        \right) },
    \\
    1- X_\text{PFR}
    &= \exp \left[
        -\alpha\, Da\,\left(
            \dfrac{k_s\,a_s\,\eta}{k_s\,a_s + k_v\,\eta}
            \right)
    \right].
\end{align}
If we define $k_r = k_s\,a_s$, Eq.~\eqref{eq: homo - CSTR} and \eqref{eq: homo - PFR} could be written as
\begin{align}
    1- X_\text{CSTR}
    &= \frac{1}{1+\alpha\,Da},
    \\
    1- X_\text{PFR}
    &= \exp (-\alpha\, Da),
\end{align}
which have the same form in \citet{li2011catalytic} and \citet{wang2015performance}, as well as in \citet{liujs2016}.

Comparing the four equations above, 
\begin{equation}
    \left(
    \dfrac{k_s\,a_s\,\eta}{k_s\,a_s + k_v\,\eta}
    \right)
    \label{eq: effection of i/d diffusion}
\end{equation}
is the only different due to the consideration of internal and external mass transfer. 
This term can be calculated. 
If we define an apparent reaction constant, 
\begin{equation}
    k_r = k_v\,
    \left(
    \dfrac{k_s\,a_s\,\eta}{k_s\,a_s + k_v\,\eta}
    \right). 
\end{equation}
The expressions of homogeneous and heterogeneous will be the same. 

\section{Conclusion}
In the works of Li and Wang, as well as Liu, the internal diffusion and external diffusion are combined with the reaction constant based on the volume of catalyst to define a new apparent reaction constant. 
By dividing Eq.~\eqref{eq: effection of i/d diffusion}, the real reaction constant could be calculated.

\paragraph{DATA: July 23, 2019. }

% --------------------------------------------------------------------
%
% --------------------------------------------------------------------
