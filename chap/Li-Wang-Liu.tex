%
%
%
\chapter{Hydrodynamics and Reaction Performances in Circulating Fluidized Bed}
\section{General Contents}

\section{Experimental Details}

\section{Hydrodynamics}

\section{Reaction Performances: Ozone Decomposition}



% -----------------------------------------------------------------
% something old
% -----------------------------------------------------------------
\section[Li, Dongbing, 2010]{Thesis, \citep{li2010}}
\textit{Investigation of circulating fluidized bed riser and downer reactor performance for catalytic ozone decomposition.}

Dimension: rise (76 mm i.d., 10.2 m high), downer (76 mm i.d., 5.8 m high). Superficial ags velocity $(U_g)$: $2-5\,\si{m/s}$. 
Solids circulation rate $(G_s)$: $50-100\,\si{kg/(m^2\,s)}$.

In this thesis, the hydrodynamics, ozone decomposition in the riser and downer were studied; 
the performances were compared between riser and downer; 
the ozone distributions in different angular position in riser were studied.

\subsection[Ozone Decomposition In Downer, 2011]{Ozone Decomposition In Downer, \citep{li2011catalytic}}
\textit{Catalytic reaction in a circulating fluidized bed downer: Ozone decomposition.}
\subsubsection{Experimental}
\paragraph{Setups. }
\begin{enumerate}
    \item Dimension: rise (76 mm i.d., 10.2 m high), 
    downer (51 mm i.d., 4.9 m high; 76 mm i.d., 5.8 m high), 
    downcomer (203 mm i.d.). 

    \item Superficial gas velocity $(U_g)$: $2,\,3,\,5\,\si{m/s}$. 
    Solids circulation rate $(G_s)$: $50$ and $100\,\si{kg/(m^2\,s)}$.
    
    \item Air feeding: $T = 22 \cels$, $\mathrm{relative\, humidity} = 19\%$; 
    controlled by a gate valve and measured by a rotameter.
    
    \item Solid feeding: controlled by a ball valve.
    
    \item Gas distribution: aluminum plates ($9.5\,\si{mm}\times 30$ holes and 47$\%$ voids).
    
    \item $\ce{O_3} / \ce{O_2}$ feeding: the ozone and oxygen mixture was mixed with air before entering in the downer. 
    The initial ozone concentration ($C_0$) is 10 to 25 ppm.
    
    \item UV ozone analyzer: Model 49i, Thermo Electron Inc.
    
    \item Local solid holdup measurement: PV-5 optical fiber probes.
    
    \item Sampling positions: shown in the Table~\ref{tab: sampling positions}.
\end{enumerate}

%sampling positions
\begin{table}[!h]
    \small
    % \centering
    \flushright
    \caption{Sampling positions}
    \label{tab: sampling positions}
    \begin{tabular}{lccccccccc}
        \toprule
        Axial ($x/\si{m}$) & 0.22 & 0.61 & 1.12 & 1.63 & 2.13 & 2.64 & 3.26 & 4.02 & 4.99 \\
        Radial ($r/R$) & 0 & 0.316 & 0.548 & 0.707 & 0.837 & 0.949 & & &  \\
        \bottomrule
    \end{tabular}
\end{table}

\paragraph{Preparing the FCC particles. }
The fresh FCC particles were
1) submerged into 20 wt$\%$ $\ce{Fe(NO3)2}$ solution;
2) dried at ambient temperature;
3) calcined at $500\,\cels$ for 10 hours;
4) broken by blender (agglomerates);
5) sited by sieve.

Particle size would affect the hydrodynamics largely. 
By passing through the riser, the fines ($\leq 10\,\si{\mu m}$) could be shifting. 
After eight weeks, $d_p = 62.192 \pm 1.166 \,\si{\mu m}$.

Properties:
1) the iron content in the particles increased from $0.47\%$ to $1.44\%$;
2) the BET surface area of particles decreased from $261\,\si{m^2/g}$ to $230\,\si{m^2/g}$;
3) Sauter mean particle size: $60\,\si{\mu m}$;
4) apparent particle density: $\rho_p = 1370\,\si{kg/m^3}$;
5) bulk particle density: $\rho_b = 795\,\si{kg/m^3}$.

\paragraph{Initial kinetic studies. }
In the fixed bed, the activation energy in Arrhenius is $89\, \si{kJ/m^3}$.

\subsection{Hydrodynamics in the downer}
Which are summarized in the Table~\ref{tab: hydrodynamics in downer}. 
% Summary of hydrodynamics in downer
\begin{table}[htb]
    \small
    % \centering
    \flushright
    \caption{Summary of hydrodynamics in downer}
    \begin{tabular}{lp{12em}p{9em}}
    \toprule
             & \multicolumn{1}{l}{Radial} 
             & Axial \\
    \midrule
    Holdup   & 0 - 0.6: uniform\newline{}0.6 - 1: increase 
             & \multicolumn{1}{p{12.085em}}{0 - 2 m: decrease\newline{}2 - 5 m: level off} \\
    Velocity & \multicolumn{1}{l}{almost uniform} 
             & increase to steady velocity \\
    Flux     & $U_g = 5 \,\si{m/s}$: uniform\newline{}low $U_g$: high $F_s$ in the wall region 
             & slightly higher at entrance region \\
    \bottomrule
    \end{tabular}%
    \label{tab: hydrodynamics in downer}%
\end{table}%
  
\subsection{Ozone concentration in the downer}
The positions for measurements are the same as those for hydrodynamics.
\paragraph{Overview. }
In the axial direction, the concentration decreased along the positive direction. 
In the radial direction, the profile changed from uniform to parabolic with increasing of the axial position.

\paragraph{Radial profiles. }
There was a high correlation between the ozone concentration and solids holdup. 
Local radial gradients calculated by the Equation~\eqref{eq: local radial gradients} showed the ozone distributions. 
The cross-sectional average radial gradient, $\overline{\mathrm{grad}(C/C_0)}$, was calculated by trapezoidal rule with the local values.
\begin{equation}
    \mathrm{grad} (C/C_0) 
    = \frac{\partial \left[(C/C_0)/\overline{(C/C_0)} \right]}{\partial (r/R)}
    \label{eq: local radial gradients}
\end{equation}
The results showed the distributions in the entrance and exit region ($z < 0.5\,\si{m}$ or $z > 4.5\,\si{m}$) were uniform. 
In the middle region, the non-uniformity was intensified and then decreased.
The gradients decreased with the increasing solids circulation rate.

\paragraph{Axial profiles. }
The conversion was high in the inlet region and low in the exit region.
The profiles could be predicted by the ideal PFR model, which was shown in the Equation~\eqref{eq: PFR axial profiles}.
\begin{equation}
    \frac{C}{C_0} 
    = \exp\left[ 
        -\frac{k_r\,\bar{\varepsilon}_s\, (1-\bar{\varepsilon}_s)\,H}{U_g} 
        \right]
    \label{eq: PFR axial profiles}
\end{equation}
where $k_r$ is the apparent reaction rate constant. 
In the inlet region, the conversion was higher than the prediction. 
However, the overall conversion was lower due to performance in the the exit region.
Strong correlation between holdup and conversion.

\paragraph{Gas-solid contact efficiency. }
For first-order reaction,
\begin{equation}
    \frac{C}{C_0} 
    = \exp\left[ 
        -\frac{\alpha\,k_r\,\bar{\varepsilon}_s\, (1-\bar{\varepsilon}_s)\,H}{U_g} 
        \right]
    = \exp\left[ -\alpha\,k'_r \right]
\end{equation}
where $\alpha$ is the contact efficiency and $k'_r$ is the Damkoehler number.

``The contact efficiency represents the fraction of the external surface area of the catalysts available for the diffused reactant from the gas phase. 
It can also be regarded as the utilization efficiency of catalysts in the CFB reactor compared to that in the plug flow reactor.''



%
%
%----------------------------------------------------------------//
\section[Wang, Chengxiu, 2013]{Wang, Chengxiu, 2013, \citep{wangcx2013}}
\textit{High density gas-solids circulating fluidized bed riser and downer reactors.}

Dimension is the same as the Li, Dongbing's: rise (76 mm i.d., 10.2 m high), downer (76 mm i.d., 5.8 m high). 
Superficial ags velocity $(U_g)$: $3-9\,\si{m/s}$. 
Solids circulation rate $(G_s)$: $100-1000\,\si{kg/(m^2\,s)}$. 
The holdup was much larger than Li's.

Similarly, in this thesis, the hydrodynamics and ozone decomposition in the riser and downer were studied;
the performances were compared between riser and downer; 
the ozone distributions in different angular position in riser were studied.

%
%
%----------------------------------------------------------------//
\section[Liu, Jiangshan, 2016]{Liu, Jiangshan, 2016, \citep{liujs2016}}
\textit{Reactor Performances and Hydrodynamics of Various Gas-Solids Fluidized Beds.}