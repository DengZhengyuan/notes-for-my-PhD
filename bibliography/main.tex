\documentclass[12pt]{report}
\usepackage[top=1in, bottom=1in, left=1in, right=1in]{geometry}
\usepackage{graphicx}
\usepackage{grffile}
\usepackage{longtable}
\usepackage{wrapfig}
\usepackage{rotating}
\usepackage[normalem]{ulem}
\usepackage{textcomp}
% --------------- math --------------- %
\usepackage{amssymb}
\usepackage{amsmath}
\usepackage{siunitx}
% --------------- math --------------- %
\usepackage{bm}
\usepackage{capt-of}
\usepackage{hyperref}
\usepackage{booktabs}
\usepackage{titlesec}
\usepackage{multirow}
\usepackage{listings}
\usepackage{flafter}
\usepackage{ntheorem}
\usepackage{subfigure}
\usepackage{indentfirst}
\usepackage{appendix}
\usepackage{float}
% --------------- citing --------------- %
\usepackage{apacite}
% \usepackage{cite}
\usepackage{natbib}
% --------------- citing --------------- %
\usepackage{url}
% \usepackage{mathspec}
\usepackage{caption}
\usepackage{cancel}
\usepackage{times}
\usepackage{mathptmx}
\usepackage{xeCJK}
\usepackage{threeparttable}
% \usepackage{mhchem}

% -----------------------------------------

%---------------------------------------------%

\newcommand{\dint}{\displaystyle\int}
\newcommand{\ig}{\mathrm{ig}}
\def\cels{\ensuremath{^\circ\hspace{-0.09em}\mathrm{C}}}

\newcommand{\tabincell}[2]{\begin{tabular}{@{}#1@{}}#2\end{tabular}}
\newcommand{\etal}{\textit{et al}}
\newcommand{\ox}{\mathrm{O_2}}
\newcommand{\oz}{\mathrm{O_3}}

\setlength\jot{10pt}

\setmainfont{Times New Roman}

\captionsetup{font={sf}}

% 正文行距
\linespread{1.5}
\setlength{\parskip}{4pt}
% 首行缩进
\setlength{\parindent}{2em}

\title{Anotated bibliography}
\author{Zhengyuan Deng}
\date{\today}

%----------------------------------------------------------------%
%                         Begin the paper                        %
%----------------------------------------------------------------%

\begin{document}
\maketitle

\newpage
\tableofcontents
% \listoffigures
% \listoftables
%----------------------------------------------------------------%
%                                                                %
%----------------------------------------------------------------%
\chapter{Thesises}
\section{Experiments}
%
%
%----------------------------------------------------------------//
\subsection[Li, Dongbing, 2010]{Li, Dongbing, 2010, \cite{li2010}}
\textit{Investigation of circulating fluidized bed riser and downer reactor performancefor catalytic ozone decomposition.}

Dimension: rise (76 mm i.d., 10.2 m high), downer (76 mm i.d., 5.8 m high). Superficial ags velocity $(U_g)$: $2-5\,\si{m/s}$. 
Solids circulation rate $(G_s)$: $50-100\,\si{kg/(m^2\,s)}$.

In this thesis, the hydrodynamics, ozone decomposition in the riser and downer were studied; 
the performances were compared between riser and downer; 
the ozone distributions in different angular position in riser were studied.

%
%
%----------------------------------------------------------------//
\subsection[Wang, Chengxiu, 2013]{Wang, Chengxiu, 2013, \cite{wangcx2013}}
\textit{High density gas-solids circulating fluidized bed riser and downer reactors.}

Dimension is the same as the Li, Dongbing's: rise (76 mm i.d., 10.2 m high), downer (76 mm i.d., 5.8 m high). 
Superficial ags velocity $(U_g)$: $3-9\,\si{m/s}$. 
Solids circulation rate $(G_s)$: $100-1000\,\si{kg/(m^2\,s)}$. 
The holdup was much larger than Li's.

Similarly, in this thesis, the hydrodynamics and ozone decomposition in the riser and downer were studied;
the performances were compared between riser and downer; 
the ozone distributions in different angular position in riser were studied.

\section{Simulations}
%
%
%----------------------------------------------------------------//
\subsection[Kong, Lei, 2012]{Kong, Lei, 2012, \cite{konglei2012}}
\textit{Numerical simulation of catalytic ozone decomposition reaction in a gas-solids circulating fluidized bed riser.}

Based on the Li's experiments, the numerical simulations were studied in this thesis. 
The two-fluid model (granular kinetic theory) was adopted in the research. 
The effect of the wall boundary conditions were studies. 
The radial and axial profiles of the ozone composition were simulated and compared with the experimental results in Li's.

%
%
%----------------------------------------------------------------//
\subsection[Liu, Yunfeng, 2018]{Liu, Yunfeng, 2018, \cite{liuyf2018}}
\textit{Numerical simulation of three-phase flows in the inverse fluidized bed.}

\chapter{Articles}
\section{Review}
%
%
%----------------------------------------------------------------//
\subsection[Ge, Wei, 2019]{Ge, Wei, 2019, \cite{ge2019multiscale}}
\textit{Multiscale structures in particle--fluid systems: Characterization, modeling, and simulation.}

This is a very general paper that focused on the mesoscale structure in the particle-fluid system, or the clusters in the fluidized bed. 
The paper discussed not only the experimental researches but also the modeling studies. 
%
%
%----------------------------------------------------------------//
\subsection[Zhong, Wenqi, 2016]{Zhong, Wenqi, 2016, \cite{zhong2016cfd}}
\textit{CFD simulation of dense particulate reaction system: Approaches, recent advances and applications.}

This paper focused on the modeling works in the dense particulate reaction systems. 
Recent advances and applications of the Eulerian-Eulerian approach and the Eulerian-Lagrangian approaches, as well as the limitations.

\section{Modeling}
%
%
%----------------------------------------------------------------//
\subsection[Gidaspow, kinetic theory for bubbling, 1990]{Gidaspow, kinetic theory for bubbling, 1990, \cite{Ding1990}}
\textit{A bubbling fluidization model using kinetic theory of granular flow.}

This is the first time to use the kinetic theory of granular flow to simulate the fluidized bed. 
In this paper, a two dimensional bubbling bed was investigated. 

%
%
%----------------------------------------------------------------//
\subsection[Gidaspow, kinetic theory for riser, 2003]{Gidaspow, kinetic theory for riser, 2003, \cite{Huilin2003}}
\textit{Hydrodynamic simulation of gas-solid flow in a riser using kinetic theory of granular flow.}

\section{Grid Study}
%
%
%----------------------------------------------------------------//
\subsection[Grid study, 2014]{Grid study, \cite{Li2014}}
\textit{Reprint of ``CFD simulations of circulating fluidized bed risers, part I: Grid study''.}

%
%
%----------------------------------------------------------------//
\subsection[Inlet profile study, 2010]{Inlet profile study, \cite{Peng2010}}
\textit{Numerical Study on the Effect of the Air Jets at the Inlet Distributor in the Gas-Solids Circulating Fluidized-Bed Risers.}

%
%
%----------------------------------------------------------------//
\subsection[Grid study, 2005]{Grid study, \cite{AndrewsIV2005}}
\textit{Coarse-grid simulation of gas-particle flows in vertical risers.}

%
%
%----------------------------------------------------------------//
\subsection[Grid study, 2001]{Grid study, \cite{Agrawal2001}}
\textit{The role of meso-scale structures in rapid gas–solid flows.}

%
%
%----------------------------------------------------------------//
\subsection[Grid study, 2001]{Grid study, \cite{Zhang2001}}
\textit{High-resolution three-dimensional numerical simulation of a circulating fluidized bed.}



\section{Ozone Decomposition}
%
%
%----------------------------------------------------------------//
\subsection[2003]{\cite{Therdthianwong2003}}
\textit{Modeling and simulation of circulating fluidized bed reactor with catalytic ozone decomposition reaction.}

%
%
%----------------------------------------------------------------//
\subsection[2004]{\cite{Hansen2004}}
\textit{A three-dimensional simulation of gas/particle flow and ozone decomposition in the riser of a circulating fluidized bed.}

%
%
%----------------------------------------------------------------//
\subsection[2008]{\cite{Dong2008}}
\textit{A multiscale mass transfer model for gas-solid riser flows: Part II-Sub-grid simulation of ozone decomposition.} 



\chapter{Books}
\section{Modeling}
%
%
%----------------------------------------------------------------//
\subsection[Computational methods for multiphase flow]{Computational methods for multiphase flow, \cite{prosperetti2009computational}}
\textit{Computational methods for multiphase flow.}

This book needs the graduate-level fluid mechanics and numerical methods knowledge as the background. 
The first part introduced the methods for solving the N-S equations by finite difference and finite element methods, which includes computational approach, grid methods for particles, and finite element methods for particulate flows.
The second part introduced the particles models and two-fluid models, which includes point-particle methods, two-fluid methods, and coupled methods. 
Also, the Lattice Boltzmann model was included. 

\section{Fluid Mechanics}
%
%
%----------------------------------------------------------------//
\subsection[A first course in continuum mechanics]{A first course in continuum mechanics, \cite{gonzalez2008first}}
\textit{A first course in continuum mechanics.}

This book is a background material for mathematics. 
The first part introduces the tensor algebra and calculus in three-dimensional Euclidean space. 
For this, \cite{matthews2012vector} can also support. 
The second part introduces the basic equations refer to the fluid mechanics, which includes mass, motion, momentum, energy, and entropy. The third part is not necessary.

The calculus and differential equations should be considered as the fundamental materials.

After this, \cite{wang2014fluid} and \cite{kundu2012fluid} can also support the knowledge for fluid mechanics.

\section{Manual}
%
%
%----------------------------------------------------------------//
\subsection[ANSYS Fluent UDF Manual]{ANSYS Fluent UDF Manual, \cite{udfmanual}}
\textit{ANSYS Fluent UDF Manual.}

The manual for user-defined functions in the ANSYS Fluent. 
UDF is necessary for modifying the inlet profiles and coupling chemical reactions with the fluid models in the software.
%
%
%----------------------------------------------------------------//
\subsection[ANSYS Fluent Theory Guide]{ANSYS Fluent Theory Guide, \cite{FluentTheoryGuide}}
\textit{ANSYS Fluent Theory Guide.}

This manual talks about the theoretical information about the models used in the ANSYS Fluent.

%
%
%----------------------------------------------------------------//
\subsection[C Primer Plus]{C Primer Plus, \cite{Prata:2013:CPP:2578965}}
\textit{C Primer Plus.}

A complete tutorial on C programming.

%---------------------------------------------%
\newpage
\bibliographystyle{apacite}
\bibliography{bib}

\end{document}