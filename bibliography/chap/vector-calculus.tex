\chapter{Vector Calculus}
This chapter records the notes for vector calculus from a book by \citet{matthews1998vector}.

\section{Vector Algebra}
\subsection{Dot product}
The dot product or scalar product of two vectors is a scalar quantity.
\begin{equation}
    \bm{a} \cdot \bm{b} = |\bm{a}| |\bm{b}| \cos \theta
\end{equation}
\begin{itemize}
    \item $ \bm{a}\cdot \bm{b} = \bm{b}\cdot \bm{a} $.
    \item The quantity $|\bm{b}|\cos \theta$ represents the component of the vector $\bm{b}$ in the direction of the vector $\bm{a}$.
    \item $ \bm{a}\cdot \bm{b} = a_1 b_1 + a_2 b_2 + a_3 b_3 $.
    \item $\bm{e}_1 \cdot \bm{e}_2 = 0$. 
\end{itemize}
    
\subsection{Cross product}
The cross product or vector product of two vectors is a vector quantity. 
The magnitude is $|\bm{a}| |\bm{b}| \sin \theta$, and the direction is perpendicular to them in a right-handed sense. 
Then,
\begin{equation}
    \bm{a} \times \bm{b} 
    = |\bm{a}| |\bm{b}| \sin \theta\,\bm{u}
    =   
        \begin{vmatrix}
            \bm{e}_1 & \bm{e}_2 & \bm{e}_3 \\
            \bm{a}_1 & \bm{a}_2 & \bm{a}_3 \\
            \bm{b}_1 & \bm{b}_2 & \bm{b}_3 
        \end{vmatrix}
\end{equation}
where $\bm{u}$ is a unit vector perpendicular to them in a right-handed sense. 
\begin{itemize}
    \item $\bm{a} \times \bm{b} = - \bm{b} \times \bm{a}$.
    \item $\bm{e}_1 \times \bm{e}_2 = 1$. 
\end{itemize}

\subsection{Scalar triple product}
The scalar triple product is defined to be 
\begin{equation}
    \bm{a} \cdot \bm{b} \times \bm{c}
    =
    \bm{a} \cdot (\bm{b} \times \bm{c})
    =
        \begin{vmatrix}
            \bm{a}_1 & \bm{a}_2 & \bm{a}_3 \\
            \bm{b}_1 & \bm{b}_2 & \bm{b}_3 \\
            \bm{c}_1 & \bm{c}_2 & \bm{c}_3 
        \end{vmatrix}
    ,
\end{equation}
written $[\bm{a},\;\bm{b},\;\bm{c}]$. The brackets are unnecessary.

\begin{itemize}
    \item If any two of the vectors are equal, the scalar triple product is zero.
    \item $ \bm{a} \cdot \bm{b} \times \bm{c}
            = \bm{a} \times \bm{b} \cdot \bm{c} $.
    \item $ \bm{a} \cdot \bm{b} \times \bm{c}
            = \bm{b} \cdot \bm{c} \times \bm{a} 
            = \bm{c} \cdot \bm{a} \times \bm{b} $.
\end{itemize}

\subsection{Vector triple product}
The vector triple product is 
$\bm{a} \times (\bm{b}\times \bm{c})$. 
The brackets are important. 
It can expanded as
\begin{equation}
    \bm{a} \times (\bm{b}\times \bm{c})
    = (\bm{a} \cdot \bm{c}) \bm{b}
    - (\bm{a} \cdot \bm{b}) \bm{c}.
\end{equation}
More information see page 16.

\section{Line, Surface, and Volume Integrals}
\subsection{Line integrals}
\begin{example}
    A particle moves along a curve path $C$ by force $\bm{F}(\bm{r})$. 
    $\bm{r}$ is the position vector, $\bm{r} = (x,\;y,\;z)$. 
    What is the total amount of work?
    \begin{equation}
        \lim_{N\to \infty} \sum_{i=1}^N \bm{F}_i \cdot \bm{dr}_i
        = \int_C \bm{F} \cdot \bm{dr}.
    \end{equation}
    \qed
\end{example}
\paragraph{Evaluation.}
Line integrals are evaluated by using a parameter, time ($t$), together with a formula giving the value of the position vector $\bm{r}$ in terms of $t$.
\begin{equation}
    \int_C \bm{F} \cdot \bm{dr}
    = \int \bm{F} \cdot \frac{\bm{dr}}{dt} dt 
\end{equation}
where 
\begin{equation}
    \frac{\bm{dr}}{dt} 
    = \left( \frac{dx}{dt},\; \frac{dy}{dt},\; \frac{dz}{dt} \right).
\end{equation}

Line integrals sometimes occur over curves that are close. 
In this case, the integral is written using the symbol $\oint$. 

\paragraph{Conservative vector fields. }
\begin{definition}
    Conservative vector fields. 
    
    A vector field $\bm{F}$ is said to be conservative if it has the property that the line integral of $\bm{F}$ around any closed curve $C$ is zero:
    \begin{equation}
        \oint_C \bm{F} \cdot \bm{dr} = 0.
    \end{equation}
    \qed
\end{definition}
The line integral of $\bm{F}$ along a curve only depends on the endpoints of the curve, not on the path taken by the curve.

\paragraph{Other forms of line integrals. }
\begin{equation}
    \int_C \phi \bm{dr}
    \quad \mathrm{and} \quad
    \int_C \bm{F} \times \bm{dr}. 
\end{equation}

\subsection{Surface integrals}
\begin{example}
    Suppose that fluid flows with velocity $\bm{u}(\bm{r},t)$ through a pipe with $A$ cross-sectional area. 
    What is the total volume of fluid passing through the pipe per unit time?
    
    Suppose the first case: the velocity is directed parallel to the walls of the pipe, with speed $|\bm{u}| = U_0$. 
    In this case, the fluid moves along the pipe as if it were a solid block. 
    In a time $t$, the fluid moves a distance $U_0 t$, so the volume is $U_0 t A$. 
    Thus, the flow rate, or flux is 
    \begin{equation}
        Q = \frac{U_0 t A}{t} = U_0 A.  
    \end{equation}

    Suppose the second case: the velocity is again directed parallel to the walls, but the speed depends on the position within the pipe: 
    $|\bm{u}| = U_0 (x,y)$.
    Besides, the pipe has a square cross-section. 
    Then, the surface element $dS = dx \, dy$. 
    Thus, 
    \begin{align}
        dQ &= U_0(x,y) \, dS = U_0(x,y) \, dx\,dy, \\
        \Longrightarrow \quad
        Q &= \iint_S U_0(x,y) \, dx\,dy. 
    \end{align}

    In the case 1 and 2, the fluid flow direction is perpendicular to the surface. 
    The third case where the vector field $\bm{u}$ and the surface $S$ are both arbitrary. 
    Therefore, only the component of $\bm{u}$ perpendicular to $dS$ contributes to the flux across $dS$.
    Here, it is necessary to introduce a normal vector $\bm{n}$ to the surface $dS$.
    \textbf{The component of $\bm{u}$ perpendicular to $dS$ is then the component of $\bm{u}$ in the direction of $\bm{n}$, which is just $\bm{u} \cdot \bm{n}$.} 
    Therefore, 
    \begin{align}
        dQ &= \bm{u} \cdot \bm{n} \, dS, \\
        \Longrightarrow \quad
        Q &= \iint_S \bm{u} \cdot \bm{n} \, dS. 
    \end{align}

    If the surface is closed, the integral could be written as 
    \begin{equation}
        \oiint_S \bm{u} \cdot \bm{n} \, dS.
    \end{equation}

    \qed 
\end{example}

\paragraph{Evaluation. }
For the second case, if  the square surface given by 
$0 \leq x \leq 1, \; 0 \leq y \leq 1$.
\begin{equation}
    Q = \iint_S U_0(x,y) \, dx\,dy 
      = \int^1_0 \int^1_0 U_0(x,y) \, dx\,dy.
\end{equation}

For the third case, the surface $S$ is curved and can be written in terms of two parameters, $v$ and $w$, so the position vector $\bm{r} = \bm{r}(v,w)$. 
To evaluate the surface integral we need an expression for $\bm{n} \, dS$. 
The cross product of two vectors gives a vector perpendicular to both an with a magnitude equal to the area of the parallelogram create by two vectors. 
Thus, 
\begin{equation}
    \iint_S \bm{u} \cdot \bm{n} \, dS
    = \iint_S \bm{u} \cdot 
    \frac{\partial \bm{r}}{\partial v} \times \frac{\partial \bm{r}}{\partial w} \, dv \, dw.
\end{equation}

\subsection{Volume integrals}
\begin{example}
    Suppose that an object of volume $V$ has a density $\rho$, and the density is a function of position, $\rho = \rho (\bm{r})$.
    What is the total mass ($M$) of the object?
    \begin{equation}
        \iiint_V \rho dV 
        = \lim_{N\to \infty} \sum^N_{i=1} \rho(\bm{r}_i)\,\delta V_i.
    \end{equation}

    Volume integrals can also be used to compute the volumes of objects, in which case $\rho =1$.
    \begin{equation}
        \iiint_V \bm{u} \, dV
    \end{equation}
    where $\bm{u}$ is a vector field.
\end{example}

\section{Partial Differentiation and Taylor Series}
\subsection{Taylor series in more than one variable}
