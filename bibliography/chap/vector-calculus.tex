\chapter{Vector Calculus}
This chapter records the notes for vector calculus from a book by \citet{matthews1998vector}.

\section{Vector Algebra}
\subsection{Dot product}
The dot product or scalar product of two vectors is a scalar quantity.
\begin{equation}
    \bm{a} \cdot \bm{b} = |\bm{a}| |\bm{b}| \cos \theta
\end{equation}
\begin{itemize}
    \item $ \bm{a}\cdot \bm{b} = \bm{b}\cdot \bm{a} $.
    \item The quantity $|\bm{b}|\cos \theta$ represents the component of the vector $\bm{b}$ in the direction of the vector $\bm{a}$.
    \item $ \bm{a}\cdot \bm{b} = a_1 b_1 + a_2 b_2 + a_3 b_3 $.
    \item $\bm{e}_1 \cdot \bm{e}_2 = 0$. 
\end{itemize}
    
\subsection{Cross product}
The cross product or vector product of two vectors is a vector quantity. 
The magnitude is $|\bm{a}| |\bm{b}| \sin \theta$, and the direction is perpendicular to them in a right-handed sense. 
Then,
\begin{equation}
    \bm{a} \times \bm{b} 
    = |\bm{a}| |\bm{b}| \sin \theta\,\bm{u}
    =   
        \begin{vmatrix}
            \bm{e}_1 & \bm{e}_2 & \bm{e}_3 \\
            \bm{a}_1 & \bm{a}_2 & \bm{a}_3 \\
            \bm{b}_1 & \bm{b}_2 & \bm{b}_3 
        \end{vmatrix}
\end{equation}
where $\bm{u}$ is a unit vector perpendicular to them in a right-handed sense. 
\begin{itemize}
    \item $\bm{a} \times \bm{b} = - \bm{b} \times \bm{a}$.
    \item $\bm{e}_1 \times \bm{e}_2 = 1$. 
\end{itemize}

\subsection{Scalar triple product}
The scalar triple product is defined to be 
\begin{equation}
    \bm{a} \cdot (\bm{b} \times \bm{c})
    =
        \begin{vmatrix}
            \bm{a}_1 & \bm{a}_2 & \bm{a}_3 \\
            \bm{b}_1 & \bm{b}_2 & \bm{b}_3 \\
            \bm{c}_1 & \bm{c}_2 & \bm{c}_3 
        \end{vmatrix}
    ,
\end{equation}
written $[\bm{a},\;\bm{b},\;\bm{c}]$. The brackets are unnecessary.

\begin{itemize}
    \item If any two of the vectors are equal, the scalar triple product is zero.
    \item $ \bm{a} \cdot \bm{b} \times \bm{c}
            = \bm{a} \times \bm{b} \cdot \bm{c} $.
    \item $ \bm{a} \cdot \bm{b} \times \bm{c}
            = \bm{b} \cdot \bm{c} \times \bm{a} 
            = \bm{c} \cdot \bm{a} \times \bm{b} $.
\end{itemize}

\subsection{Vector triple product}
The vector triple product is 
$\bm{a} \times (\bm{b}\times \bm{c})$. 
The brackets are important. 
It can expanded as
\begin{equation}
    \bm{a} \times (\bm{b}\times \bm{c})
    = (\bm{a} \cdot \bm{c}) \bm{b}
    - (\bm{a} \cdot \bm{b}) \bm{c}.
\end{equation}
More information see page 16.

\section{Line, Surface, and Volume Integrals}
\subsection{Line integrals}
\begin{example}
    A particle moves along a curve path $C$ by force $\bm{F}(\bm{r})$. 
    What is the total amount of work?
    \begin{equation}
        \lim_{N\to \infty} \sum_{i=1}^N \bm{F}_i \cdot \bm{dr}_i
        = \int_C \bm{F} \cdot \bm{dr}.
    \end{equation}
    \qed
\end{example}

