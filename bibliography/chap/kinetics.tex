\chapter{Kinetics}
% \section{Kinetics in ozone decomposition}

% \section{Kinetics in fluidized bed}

% \section{Pseudo kinetics}

\section{Reaction Models in CFB}
The models are based on CSTR, PFR with internal and external mass transfer from \citet{jiang1991baffle}.

\subsection{Original CSTR and PFR models}
\paragraph{CSTR. }
Balance in mole flow:
$[\si{mol/s}]$
\begin{equation*}
    \mathrm{enters}-\mathrm{leaves}=\mathrm{accumulations}
\end{equation*}
\begin{equation}
    U_g C_i A - U_g C_o A = V_t \, \frac{dC}{dt}
    \label{eq: kinetics - original CSTR}
\end{equation}
where
\begin{itemize}
    \item $U_g$ $[\si{m/s}]$ is the superficial gas velocity;
    \item $C$ $[\si{mol/m^3}]$ is the reactant concentration, and $i$, $o$ represent inlet and outlet, respectively;
    \item $A$ $[\si{m^2}]$ is the cross-sectional area of the reactor;
    \item $V_t$ $[\si{m^3}]$ is the total volume of the reactor.
\end{itemize}

\paragraph{PFR. }
Balance in mole flow:
$[\si{mol/s}]$
\begin{equation*}
    \mathrm{enters}-\mathrm{leaves}=\mathrm{accumulations}
\end{equation*}
\begin{align}
    U_g A C|_z - U_g A C|_{z+\Delta z}
     & = \Delta V \, \frac{\partial C}{\partial t} \\
    \Longrightarrow \quad
    U_g \, \frac{\partial C}{\partial z}
     & = \frac{\partial C}{\partial t}.
    \label{eq: kinetics - original PFR}
\end{align}

\subsection{CSTR and PFR models in CFB}
The ozone decomposition reaction can be considered as isothermal pseudo-first-order irreversible reaction.
In the CFB reactor, it reacts under steady state.

There are also some assumptions:
\begin{enumerate}
    \item $U_g$ is constant.
    \item Solid particles are spherical with mean diameter $d_p$.
\end{enumerate}

\subsubsection{Gas phase}
The mole balance from CSTR and PFR can be used in the gas phase of CFB.
The key point is the accumulation term, which mainly due to the external mass transfer in the gas phase.
From the Fick's law:
\begin{equation}
    \bm{J} = -D \, \nabla \bm{C}
\end{equation}
where $\bm{J}\; [\si{\sfrac{mol}{m^2\, s}}]$ is the diffusion flux: the amount of substance per unit area per unit time; 
$D \; \si{m/s}$ is the diffusivity.
\begin{align*}
    V_t\, \frac{\partial C}{\partial t} =
    \underbrace{D\,\frac{(C_g-C_s)}{\delta}}_{\text{Fick's law}}
    \underbrace{\alpha\, A_s}_{\substack{\text{available surface}\\ \text{area of particles}}} 
    \frac{V_s}{V_s}\,\frac{V_t}{V_t}
\end{align*}
where $\delta$ is the film thickness.
\begin{definition}
    $k_s,\; a_s,\; \alpha,\;\varepsilon$.
    \begin{itemize}
        \item $k_s\; [\si{\sfrac{m^3}{m^2\,s}}] = \sfrac{D}{\delta}$ is the external mass transfer coefficient.
        \item $a_s\; [\si{\sfrac{1}{m}}] = \sfrac{A_s}{V_s}$ is the surface area per unit volume of particle, for sphere is $\sfrac{6}{d_p}$.
        \item $\alpha\; [-]$ is the contact efficiency factor, which is a fraction of the external surface area of catalysts available for the diffused ozone reactant from the gas phase:
        \begin{align*}
            \frac{\text{total particle surface area}}{\text{particle surface area available to contact with ozone}}
        \end{align*}
        \item $\varepsilon_s$ is the solid holdups.
    \end{itemize}
    \qed
\end{definition}
Consequently, 
\begin{align}
    \frac{\partial C}{\partial t} =
    k_s \, a_s \, \alpha \, \varepsilon_s (C_g - C_s)
    \label{eq: kinetics - accumulation term}
\end{align}

Substitute Eq.~\eqref{eq: kinetics - accumulation term} into Eq.~\eqref{eq: kinetics - original CSTR} and \eqref{eq: kinetics - original PFR}: 
\begin{align}
    \text{CSTR: }
    & U_g\, C_{g,i} - U_g\, C_{g,o} 
    = k_s \, a_s \, \alpha \, \varepsilon_s (C_g - C_s) H \\
    \text{PFR: }
    & U_g\,\frac{\partial C}{\partial z}
    = k_s \, a_s \, \alpha \, \varepsilon_s (C_g - C_s)
\end{align}

\subsubsection{Solid phase}
\begin{align*}
    \text{external m. t.}
    = 
    \text{reaction with internal m. t.}
    \qquad \left[\si{\frac{mol}{m^3\, s}}\right]
\end{align*}
\begin{align}
    D\, \frac{(C_g - C_s)}{\delta}\, A_s
    &= \eta \, k_v \, C_s \, V_s \\
    \Longrightarrow \quad 
    k_s \, a_s (C_g - C_s) 
    &= k_v \, C_s \, \eta 
\end{align}
where $k_v\, [\si{1/s}]$ is the reaction constant based on the volume of catalyst particle; $\eta\; [-]$ is the internal mass transfer resistance.



\newpage
% --------------------------------------------------------------------
%
% --------------------------------------------------------------------

\section{Contact Efficiency}
In the work of \citet{li2011catalytic} and \citet{li2013catalytic}, the gas-solid contact efficiency, $\alpha$, is defined based on an ideal plug-flow reactor (PFR).
For the first-order reaction,
\begin{equation}
    \frac{C}{C_0}
    = \exp\left[
        -\frac{\alpha\,k_r\,\bar{\varepsilon}_s\, (1-\bar{\varepsilon}_s)\,H}{U_g}
        \right]
    = \exp\left[ -\alpha\,Da \right]
\end{equation}
where $Da$ is the Damköhler number.

\begin{definition}
    Damköhler number

    Damköhler number, $Da$, which is a dimensionless number that can give us a quick estimate of the degree of conversion that can be achieved in continuous flow reactors \citep{fogler2016element}.
    \qed
\end{definition}

In Li's work, this number represents the fraction of the external surface area of the catalysts available for the diffused reactant from the gas phase.
It can also be regarded as the utilization efficiency of catalysts in the CFB reactor compared to that in the PFR.

The overall $\alpha$ increase with the increasing $U_g$ and decreasing $G_s$ in the low density CFB downer.
Also, the $\alpha$ is less than 1 due to the negligible axial dispersion.

Since the solids holdup vary at different axial positions, it is of interest to also know how the contact $\alpha$ changes with downer elevations.
Suppose the CFB downer consist of a series of PFR.
Mass balance gives
\begin{align}
    -V_c \, dC
           & = \alpha(k_r\, \bar{\varepsilon}_{s,r} V_c) \frac{dz}{Ug/(1-\bar{\varepsilon}_{s,r})}, \\
    \Longrightarrow \quad
    \alpha & = -
    \frac{1}{(C k_r \bar{\varepsilon}_{s,r})}
    \frac{U_g}{(1-\bar{\varepsilon}_{s,r})}
    \frac{dC}{dz},
\end{align}
and $\alpha \propto - z$.
