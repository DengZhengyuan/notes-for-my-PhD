\chapter{Kinetics}
\section{Kinetics in ozone decomposition}

\section{Kinetics in fluidized bed}

\section{Pseudo kinetics}

\section{Contact efficiency}
In the work of \citet{li2011catalytic} and \citet{li2013catalytic}, the gas-solid contact efficiency, $\alpha$, is defined based on an ideal plug-flow reactor (PFR). 
For the first-order reaction,
\begin{equation}
    \frac{C}{C_0} 
    = \exp\left[ 
        -\frac{\alpha\,k_r\,\bar{\varepsilon}_s\, (1-\bar{\varepsilon}_s)\,H}{U_g} 
        \right]
    = \exp\left[ -\alpha\,Da \right]
\end{equation}
where $Da$ is the Damköhler number.

\begin{definition}
    Damköhler number

    Damköhler number, $Da$, which is a dimensionless number that can give us a quick estimate of the degree of conversion that can be achieved in continuous flow reactors \citep{fogler2016element}.
    \qed
\end{definition}

In Li's work, this number represents the fraction of the external surface area of the catalysts available for the diffused reactant from the gas phase. 
It can also be regarded as the utilization efficiency of catalysts in the CFB reactor compared to that in the PFR.  

The overall $\alpha$ increase with the increasing $U_g$ and decreasing $G_s$ in the low density CFB downer. 
Also, the $\alpha$ is less than 1 due to the negligible axial dispersion.

Since the solids holdup vary at different axial positions, it is of interest to also know how the contact $\alpha$ changes with downer elevations. 
Suppose the CFB downer consist of a series of PFR. 
Mass balance gives
\begin{align}
    -V_c \, dC 
    &= \alpha(k_r\, \bar{\varepsilon}_{s,r} V_c) \frac{dz}{Ug/(1-\bar{\varepsilon}_{s,r})}, \\
    \Longrightarrow \quad
    \alpha &= -
    \frac{1}{(C k_r \bar{\varepsilon}_{s,r})} 
    \frac{U_g}{(1-\bar{\varepsilon}_{s,r})} 
    \frac{dC}{dz},
\end{align}
and $\alpha \propto - z$. 