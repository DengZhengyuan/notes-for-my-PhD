\documentclass[fontsize=11pt, % Document font size
                             paper=letter, % Document paper type
                             twoside, % Shifts odd pages to the left for easier reading when printed, can be changed to oneside
                             captions=tableheading,
                             index=totoc,
                             hyperref]{labbook}
 
\usepackage[bottom=10em]{geometry} % Reduces the whitespace at the bottom of the page so more text can fit

\usepackage[english]{babel} % English language
\usepackage{lipsum} % Used for inserting dummy 'Lorem ipsum' text into the template

% --------- font --------- %
\usepackage[utf8]{inputenc} % Uses the utf8 input encoding
\usepackage[T1]{fontenc} 
% Select the fonts with pdftex that will be conflict with xelatex
% \usepackage{fontspec} 
% Using the fonts in the system that should compiled by xelatex
\usepackage{textcomp}
\usepackage[osf]{mathpazo} % Palatino as the main font
\linespread{1.05}\selectfont % Palatino needs some extra spacing, here 5% extra
\usepackage[scaled=.88]{beramono} % Bera-Monospace
\usepackage[scaled=.86]{berasans} % Bera Sans-Serif
\DeclareFixedFont{\textcap}{T1}{phv}{bx}{n}{1cm} % Font for main title: Helvetica 1.5 cm
\DeclareFixedFont{\textaut}{T1}{phv}{bx}{n}{0.5cm} % Font for author name: Helvetica 0.8 cm

% ---------- figure and table ---------- %
\usepackage{graphicx} % Required for including images
\usepackage{subfigure}
\usepackage{longtable}
\usepackage{grffile} % Long caption ???
\usepackage{rotating} % Rotate the figure or table
\usepackage{caption}
\usepackage{booktabs,array} % Packages for tables

% --------------- math --------------- %
\usepackage{amssymb}
\usepackage{amsmath} % For typesetting math
\usepackage{siunitx} % SI unit
\usepackage[version=4]{mhchem} % chemical
\usepackage{bm} % bold and italia
\usepackage{cancel}
\usepackage{amsthm} % edit theorem environment
\usepackage{esint} % kinds of integral signs

\theoremstyle{definition}
\renewcommand{\qedsymbol}{$\blacksquare$}
\newtheorem{example}{Example}
\newtheorem{definition}{Definition}

% --------------- citing --------------- %
\usepackage{natbib}
% --------- coding environment --------- %
\usepackage{listings}

\usepackage{indentfirst} % Para. indent at first line
\usepackage{url}

\usepackage{etoolbox}
\usepackage[norule]{footmisc} % Removes the horizontal rule from footnotes
\usepackage{lastpage} % Counts the number of pages of the document

\usepackage[dvipsnames]{xcolor}  % Allows the definition of hex colors
\definecolor{titleblue}{rgb}{0.16,0.24,0.64} % Custom color for the title on the title page
\definecolor{linkcolor}{rgb}{0,0,0.42} % Custom color for links - dark blue at the moment

\addtokomafont{title}{\Huge\color{titleblue}} % Titles in custom blue color
\addtokomafont{chapter}{\color{OliveGreen}} % Lab dates in olive green
\addtokomafont{section}{\color{Sepia}} % Sections in sepia
\addtokomafont{pagehead}{\normalfont\sffamily\color{gray}} % Header text in gray and sans serif
\addtokomafont{caption}{\footnotesize\itshape} % Small italic font size for captions
\addtokomafont{captionlabel}{\upshape\bfseries} % Bold for caption labels
\addtokomafont{descriptionlabel}{\rmfamily}
\setcapwidth[r]{10cm} % Right align caption text
\setkomafont{footnote}{\sffamily} % Footnotes in sans serif

\deffootnote[4cm]{4cm}{1em}{\textsuperscript{\thefootnotemark}} % Indent footnotes to line up with text


\usepackage[nouppercase,headsepline]{scrpage2} % Provides headers and footers configuration
\pagestyle{scrheadings} % Print the headers and footers on all pages
\clearscrheadfoot % Clean old definitions if they exist

\automark[chapter]{chapter}
\ohead{\headmark} % Prints outer header

\setlength{\headheight}{25pt} % Makes the header take up a bit of extra space for aesthetics
\setheadsepline{.4pt} % Creates a thin rule under the header
\addtokomafont{headsepline}{\color{lightgray}} % Colors the rule under the header light gray

\ofoot[\normalfont\normalcolor{\thepage\ |\  \pageref{LastPage}}]{\normalfont\normalcolor{\thepage\ |\  \pageref{LastPage}}} % Creates an outer footer of: "current page | total pages"

% These lines make it so each new lab day directly follows the previous one i.e. does not start on a new page - comment them out to separate lab days on new pages
\makeatletter
\patchcmd{\addchap}{\if@openright\cleardoublepage\else\clearpage\fi}{\par}{}{}
\makeatother
\renewcommand*{\chapterpagestyle}{scrheadings}

% Hyperlink configuration
\usepackage[
    pdfauthor={}, % Your name for the author field in the PDF
    pdftitle={Laboratory Journal}, % PDF title
    pdfsubject={}, % PDF subject
    bookmarksopen=true,
    linktocpage=true,
    urlcolor=linkcolor, % Color of URLs
    citecolor=linkcolor, % Color of citations
    linkcolor=linkcolor, % Color of links to other pages/figures
    backref=page,
    pdfpagelabels=true,
    plainpages=false,
    colorlinks=true, % Turn off all coloring by changing this to false
    bookmarks=true,
    pdfview=FitB]{hyperref}

\usepackage[stretch=10]{microtype} % Slightly tweak font spacing for aesthetics

\setlength\parindent{2em} % Uncomment to remove all indentation from paragraphs

\newcommand{\dint}{\displaystyle\int}
\newcommand{\ig}{\mathrm{ig}}
\def\cels{\ensuremath{^\circ\hspace{-0.09em}\mathrm{C}}}

\newcommand{\tabincell}[2]{\begin{tabular}{@{}#1@{}}#2\end{tabular}}
\newcommand{\etal}{\textit{et al}}

% ----------------------------------------------------------------------------

\begin{document}

%----------------------------------------------------------------------------
%	TITLE PAGE
%----------------------------------------------------------------------------

\title{\textcap{Notes for My PhD Learning \\[1cm]  
\textaut{Beginning June, 2019}}}

\author{
    \textaut{Zhengyuan Deng}\\ \\
    PhD Student
}
\date{\today}

\maketitle % Title page

\printindex
\tableofcontents % Table of contents
\newpage % Start on a new page

\begin{addmargin}[4cm]{0cm} % Makes the text width much shorter for a compact look

\pagestyle{scrheadings} % Begin using headers

%-----------------------------------------------
%	BOOK CONTENTS
%-----------------------------------------------
\part{FLUIDIZATION}
% %
%
%
\chapter{Hydrodynamics and Reaction Performances in Circulating Fluidized Bed}
\section{General Contents}

\section{Experimental Details}

\section{Hydrodynamics}

\section{Reaction Performances: Ozone Decomposition}



% -----------------------------------------------------------------
% something old
% -----------------------------------------------------------------
\section[Li, Dongbing, 2010]{Thesis, \citep{li2010}}
\textit{Investigation of circulating fluidized bed riser and downer reactor performance for catalytic ozone decomposition.}

Dimension: rise (76 mm i.d., 10.2 m high), downer (76 mm i.d., 5.8 m high). Superficial ags velocity $(U_g)$: $2-5\,\si{m/s}$. 
Solids circulation rate $(G_s)$: $50-100\,\si{kg/(m^2\,s)}$.

In this thesis, the hydrodynamics, ozone decomposition in the riser and downer were studied; 
the performances were compared between riser and downer; 
the ozone distributions in different angular position in riser were studied.

\subsection[Ozone Decomposition In Downer, 2011]{Ozone Decomposition In Downer, \citep{li2011catalytic}}
\textit{Catalytic reaction in a circulating fluidized bed downer: Ozone decomposition.}
\subsubsection{Experimental}
\paragraph{Setups. }
\begin{enumerate}
    \item Dimension: rise (76 mm i.d., 10.2 m high), 
    downer (51 mm i.d., 4.9 m high; 76 mm i.d., 5.8 m high), 
    downcomer (203 mm i.d.). 

    \item Superficial gas velocity $(U_g)$: $2,\,3,\,5\,\si{m/s}$. 
    Solids circulation rate $(G_s)$: $50$ and $100\,\si{kg/(m^2\,s)}$.
    
    \item Air feeding: $T = 22 \cels$, $\mathrm{relative\, humidity} = 19\%$; 
    controlled by a gate valve and measured by a rotameter.
    
    \item Solid feeding: controlled by a ball valve.
    
    \item Gas distribution: aluminum plates ($9.5\,\si{mm}\times 30$ holes and 47$\%$ voids).
    
    \item $\ce{O_3} / \ce{O_2}$ feeding: the ozone and oxygen mixture was mixed with air before entering in the downer. 
    The initial ozone concentration ($C_0$) is 10 to 25 ppm.
    
    \item UV ozone analyzer: Model 49i, Thermo Electron Inc.
    
    \item Local solid holdup measurement: PV-5 optical fiber probes.
    
    \item Sampling positions: shown in the Table~\ref{tab: sampling positions}.
\end{enumerate}

%sampling positions
\begin{table}[!h]
    \small
    % \centering
    \flushright
    \caption{Sampling positions}
    \label{tab: sampling positions}
    \begin{tabular}{lccccccccc}
        \toprule
        Axial ($x/\si{m}$) & 0.22 & 0.61 & 1.12 & 1.63 & 2.13 & 2.64 & 3.26 & 4.02 & 4.99 \\
        Radial ($r/R$) & 0 & 0.316 & 0.548 & 0.707 & 0.837 & 0.949 & & &  \\
        \bottomrule
    \end{tabular}
\end{table}

\paragraph{Preparing the FCC particles. }
The fresh FCC particles were
1) submerged into 20 wt$\%$ $\ce{Fe(NO3)2}$ solution;
2) dried at ambient temperature;
3) calcined at $500\,\cels$ for 10 hours;
4) broken by blender (agglomerates);
5) sited by sieve.

Particle size would affect the hydrodynamics largely. 
By passing through the riser, the fines ($\leq 10\,\si{\mu m}$) could be shifting. 
After eight weeks, $d_p = 62.192 \pm 1.166 \,\si{\mu m}$.

Properties:
1) the iron content in the particles increased from $0.47\%$ to $1.44\%$;
2) the BET surface area of particles decreased from $261\,\si{m^2/g}$ to $230\,\si{m^2/g}$;
3) Sauter mean particle size: $60\,\si{\mu m}$;
4) apparent particle density: $\rho_p = 1370\,\si{kg/m^3}$;
5) bulk particle density: $\rho_b = 795\,\si{kg/m^3}$.

\paragraph{Initial kinetic studies. }
In the fixed bed, the activation energy in Arrhenius is $89\, \si{kJ/m^3}$.

\subsection{Hydrodynamics in the downer}
Which are summarized in the Table~\ref{tab: hydrodynamics in downer}. 
% Summary of hydrodynamics in downer
\begin{table}[htb]
    \small
    % \centering
    \flushright
    \caption{Summary of hydrodynamics in downer}
    \begin{tabular}{lp{12em}p{9em}}
    \toprule
             & \multicolumn{1}{l}{Radial} 
             & Axial \\
    \midrule
    Holdup   & 0 - 0.6: uniform\newline{}0.6 - 1: increase 
             & \multicolumn{1}{p{12.085em}}{0 - 2 m: decrease\newline{}2 - 5 m: level off} \\
    Velocity & \multicolumn{1}{l}{almost uniform} 
             & increase to steady velocity \\
    Flux     & $U_g = 5 \,\si{m/s}$: uniform\newline{}low $U_g$: high $F_s$ in the wall region 
             & slightly higher at entrance region \\
    \bottomrule
    \end{tabular}%
    \label{tab: hydrodynamics in downer}%
\end{table}%
  
\subsection{Ozone concentration in the downer}
The positions for measurements are the same as those for hydrodynamics.
\paragraph{Overview. }
In the axial direction, the concentration decreased along the positive direction. 
In the radial direction, the profile changed from uniform to parabolic with increasing of the axial position.

\paragraph{Radial profiles. }
There was a high correlation between the ozone concentration and solids holdup. 
Local radial gradients calculated by the Equation~\eqref{eq: local radial gradients} showed the ozone distributions. 
The cross-sectional average radial gradient, $\overline{\mathrm{grad}(C/C_0)}$, was calculated by trapezoidal rule with the local values.
\begin{equation}
    \mathrm{grad} (C/C_0) 
    = \frac{\partial \left[(C/C_0)/\overline{(C/C_0)} \right]}{\partial (r/R)}
    \label{eq: local radial gradients}
\end{equation}
The results showed the distributions in the entrance and exit region ($z < 0.5\,\si{m}$ or $z > 4.5\,\si{m}$) were uniform. 
In the middle region, the non-uniformity was intensified and then decreased.
The gradients decreased with the increasing solids circulation rate.

\paragraph{Axial profiles. }
The conversion was high in the inlet region and low in the exit region.
The profiles could be predicted by the ideal PFR model, which was shown in the Equation~\eqref{eq: PFR axial profiles}.
\begin{equation}
    \frac{C}{C_0} 
    = \exp\left[ 
        -\frac{k_r\,\bar{\varepsilon}_s\, (1-\bar{\varepsilon}_s)\,H}{U_g} 
        \right]
    \label{eq: PFR axial profiles}
\end{equation}
where $k_r$ is the apparent reaction rate constant. 
In the inlet region, the conversion was higher than the prediction. 
However, the overall conversion was lower due to performance in the the exit region.
Strong correlation between holdup and conversion.

\paragraph{Gas-solid contact efficiency. }
For first-order reaction,
\begin{equation}
    \frac{C}{C_0} 
    = \exp\left[ 
        -\frac{\alpha\,k_r\,\bar{\varepsilon}_s\, (1-\bar{\varepsilon}_s)\,H}{U_g} 
        \right]
    = \exp\left[ -\alpha\,k'_r \right]
\end{equation}
where $\alpha$ is the contact efficiency and $k'_r$ is the Damkoehler number.

``The contact efficiency represents the fraction of the external surface area of the catalysts available for the diffused reactant from the gas phase. 
It can also be regarded as the utilization efficiency of catalysts in the CFB reactor compared to that in the plug flow reactor.''



%
%
%----------------------------------------------------------------//
\section[Wang, Chengxiu, 2013]{Wang, Chengxiu, 2013, \citep{wangcx2013}}
\textit{High density gas-solids circulating fluidized bed riser and downer reactors.}

Dimension is the same as the Li, Dongbing's: rise (76 mm i.d., 10.2 m high), downer (76 mm i.d., 5.8 m high). 
Superficial ags velocity $(U_g)$: $3-9\,\si{m/s}$. 
Solids circulation rate $(G_s)$: $100-1000\,\si{kg/(m^2\,s)}$. 
The holdup was much larger than Li's.

Similarly, in this thesis, the hydrodynamics and ozone decomposition in the riser and downer were studied;
the performances were compared between riser and downer; 
the ozone distributions in different angular position in riser were studied.

%
%
%----------------------------------------------------------------//
\section[Liu, Jiangshan, 2016]{Liu, Jiangshan, 2016, \citep{liujs2016}}
\textit{Reactor Performances and Hydrodynamics of Various Gas-Solids Fluidized Beds.}
\chapter{Analytical Reaction Models in Fluidized Bed}
The models are based on CSTR, PFR with internal and external mass transfer. 

The ozone decomposition reaction can be considered as isothermal pseudo-first-order irreversible reaction: 
\begin{align}
    (-\mathcal{R}) = -\frac{dC}{dt} = k\,C
\end{align}
where $(-\mathcal{R})$ is chemical reaction (ozone decomposition) rate, 
and $k$ is reaction constant. 

\section{Original CSTR and PFR models}
\begin{figure}[h!]
    \raggedleft
    \includegraphics{fig/CSTR+PFR}
    \caption{Schematics of CSTR and PFR: the CSTR is on the left, and the right one is PFR; the blue lines represent the gas}
    \label{fig: CSTR + PFR} 
\end{figure}

\paragraph{CSTR. }
Balance in mole flow:
$[\si{mol/s}]$
\begin{equation*}
    \text{enters} - \text{leaves}
    = \text{accumulations} + \text{reaction}
\end{equation*}
\begin{equation}
    U_g C_i A - U_g C_o A 
    = V_t \, \cancel{\frac{dC}{dt}} 
    + V_t \,(\mathcal{-R})
    \label{eq: kinetics - original CSTR}
\end{equation}
where
\begin{itemize}
    \item $U_g$ $[\si{m/s}]$ is the superficial gas velocity;
    \item $C$ $[\si{mol/m^3}]$ is the reactant concentration, and $i$, $o$ represent inlet and outlet, respectively;
    \item $A$ $[\si{m^2}]$ is the cross-sectional area of the reactor;
    \item $V_t$ $[\si{m^3}]$ is the total volume of the reactor.
\end{itemize}

Under the steady state, the solution for the Eq.~\eqref{eq: kinetics - original CSTR} is 
\begin{equation}
    \frac{C_o}{C_i} = 1-X_\text{CSTR} 
    = \frac{1}{1+\left(\dfrac{k\,H}{U_g}\right)}
    \label{eq: solution - original - CSTR}
\end{equation}
where $k$ is reaction constant based on the volume of the reactor; 
$H$ is the length of the reactor.

\paragraph{PFR. }
Balance in mole flow:
$[\si{mol/s}]$
\begin{equation*}
    \text{enters} - \text{leaves}
    = \text{accumulations} + \text{reaction}
\end{equation*}
\begin{align}
    U_g A C|_z - U_g A C|_{z+\Delta z}
     & = \Delta V \, \cancel{\frac{\partial C}{\partial t}} 
       + \Delta V \,(\mathcal{-R})\\
    \Longrightarrow \quad
    U_g \, \frac{\partial C}{\partial z}
     & = k\,C.
    \label{eq: kinetics - original PFR}
\end{align}

Under the steady state, the solution for the Eq.~\eqref{eq: kinetics - original PFR} is 
\begin{equation}
    \frac{C_o}{C_i} = 1-X_\text{PFR} 
    = \exp{\left( -\frac{k\, H}{U_g} \right)}
    \label{eq: solution - original - PFR}
\end{equation}

\section{Homogeneous Models for Fixed Bed}
In the fixed bed, there are catalyst particles in the reactor.  
Also, the assumptions for the homogeneous models are 
\begin{enumerate}
    \item no internal and external mass transfer;
    \item the concentration of reactants and temperature at the surface of the particle is the same as the bulk gas phase.
\end{enumerate}
Under this circumstance, $k_v$ is defined as the reaction constant based on the catalyst volume. 
Then, 
\begin{equation}
    V \,(\mathcal{-R}) 
    = \underbrace{V \, \varepsilon_s}_{V_s} k_v\,C
\end{equation}
where $\varepsilon_s$ is the solid holdups, 
and $V_s$ is the volume of the solid phase. 
Therefore, the solutions for the two models in Eq.~\eqref{eq: solution - original - CSTR} and \eqref{eq: solution - original - PFR} can be written as 
\begin{align}
    \text{CSTR: }\quad
    \frac{C_o}{C_i} &= 1-X_\text{CSTR} 
    = \frac{1}
    {1+\left(\dfrac{k_v\, \varepsilon_s \, H}{U_g}\right)},
    \\
    \text{PFR: }\quad
    \frac{C_o}{C_i} &= 1-X_\text{PFR} 
    = \exp{\left( -\frac{k_v\, \varepsilon_s \, H}{U_g} \right)}.
\end{align}

The real length that gas passed is shorter than $H$ as the solid phase exists. The solutions are rewritten as 
\begin{align}
    \text{CSTR: }\quad
    \frac{C_o}{C_i} &= 1-X_\text{CSTR} 
    = \frac{1}{1+\left(\dfrac{k_v\, \varepsilon_s \,(1-\varepsilon_s) \, H}{U_g}\right)},
    \\
    \text{PFR: }\quad
    \frac{C_o}{C_i} &= 1-X_\text{PFR} 
    = \exp{\left( -\frac{k_v\, \varepsilon_s \, (1-\varepsilon_s) \, H}{U_g} \right)}.
\end{align}

\section{Heterogeneous Models}
Be different from the homogeneous models, the internal and external diffusion is considered. 
There are some assumptions: 
\begin{enumerate}
    \item $U_g$ is constant.
    \item Solid particles are spherical with mean diameter $d_p$.
\end{enumerate}

\subsection{Gas phase}
In the gas phase models, the external mass transfer should be considered. 
From the Fick's law: 
\begin{equation}
    \bm{J} = -D_m \, \nabla \bm{C}
\end{equation}
where $\bm{J}\; [\si{mol/(m^2\cdot s)}]$ is the diffusion flux: the amount of substance per unit area per unit time; 
$D_m \; [\si{m/s}]$ is the diffusivity.
\begin{align}
    \text{external mass transfer term} =
    \underbrace{D_m\,\frac{(C_g-C_s)}{\delta}}_{\text{Fick's law}}
    \,
    A_s 
    \,
    \frac{V_s}{V_s}\,\frac{V_t}{V_t}
    \qquad  \left[\frac{\si{mol}}{\cancel{\si{m^2}} \cdot \si{s}}\right]
            \,
            \cancel{\si{m^2}}
    \label{eq: external dif. - fixed bed}
\end{align}
where $\delta$ is the film thickness, 
$A_s$ is the total surface area of the catalyst.
\begin{definition}
    $k_s,\; a_s$.
    \begin{enumerate}
        \item $k_s\; [\si{m^3/(m^2\cdot s)}] = D_m/\delta$ is the external mass transfer coefficient.
        \item $a_s\; [\si{1/m}] = A_s/V_s$ is the surface area per unit volume of particle, for sphere is $6/d_p$.
    \end{enumerate}
    \qed
\end{definition}
Consequently, 
\begin{align}
    \text{R.H.S.} =
    k_s \, a_s \, \alpha \, \varepsilon_s (C_g - C_s)
    \label{eq: kinetics - accumulation term}
\end{align}

Combination of Eq.~\eqref{eq: kinetics - original CSTR}, \eqref{eq: kinetics - original PFR}, and Eq.~\eqref{eq: kinetics - accumulation term}: 
\begin{align}
    \text{CSTR: }
    & U_g\, C_{g,i} - U_g\, C_{g,o} 
    = k_s \, a_s \, \alpha \, \varepsilon_s \,(1-\varepsilon_s) (C_g - C_s) H 
    \label{eq: CSTR gas phase} \\ 
    \text{PFR: }
    & U_g\,\frac{\partial C}{\partial z}
    = k_s \, a_s \, \alpha \, \varepsilon_s \,(1-\varepsilon_s) (C_g - C_s) 
    \label{eq: PFR gas phase}
\end{align}

\subsection{Solid phase}
\begin{align*}
    \text{external diffusion}
    = 
    \text{reaction with internal diffusion}
    \qquad \left[\si{\frac{mol}{m^3\cdot s}}\right]
\end{align*}
\begin{align}
    D_m\, \frac{(C_g - C_s)}{\delta}\, A_s
    &= \eta \, k_v \, C_s \, V_s \\
    \Longrightarrow \quad 
    k_s \, a_s (C_g - C_s) 
    &= k_v \, C_s \, \eta 
    \label{eq: solid phase}
\end{align}
where $k_v\, [\si{1/s}]$ is the reaction constant based on the volume of catalyst particle; $\eta\; [-]$ is the internal mass transfer resistance.

\subsection{Rearrangement}
Rearrangement of Eq.~\eqref{eq: solid phase}:
\begin{equation}
    C_s = \frac{k_s\, a_s}{k_s\, a_s + k_v\,\eta}\, C_g
    \label{eq: rearranged solid phase}
\end{equation}

Combination of Eq.~\eqref{eq: CSTR gas phase} and \eqref{eq: PFR gas phase}, and Eq.~\eqref{eq: rearranged solid phase}:
\begin{align}
    \text{CSTR:} \quad
    & \frac{C_{g,o}}{C_{g,i}} = 1- X_\text{CSTR}
    = \frac{1}{1+ \left(
        \dfrac{k_v\,\eta\,k_s\,a_s\,\varepsilon_s\,(1-\varepsilon_s)\,H}{U_g(k_s\, a_s + k_v\,\eta)}
        \right)} 
    \label{eq: CSTR in fixed bed}\\
    \text{PFR:} \quad
    & U_g\,\frac{dC_{g}}{dz} + \frac{k_s\,a_s\,k_v\,\eta}{k_s\,a_s + k_v\,\eta}\,\varepsilon_s\,(1-\varepsilon_s)\,C_g = 0 \\
    \Longrightarrow \quad 
    & \frac{C_{g,o}}{C_{g,i}} = 1- X_\text{PFR}
    = \exp \left[
        - \dfrac{k_v\,\eta\,k_s\,a_s\,\varepsilon_s\,(1-\varepsilon_s)\,H}{U_g(k_s\, a_s + k_v\,\eta)}  
    \right]
    \label{eq: PFR in fixed bed}
\end{align}

\subsection{Calculation}
In the models above, the external diffusion coefficient and the effectiveness factor due to internal diffusion need to be obtained.

For external diffusion coefficient \citep[pg. 691]{fogler2016element}, $k_s$, 
\begin{align}
    Sh &= \frac{k_s\, d_p}{D_m} 
        = 2.0 + 0.63 Re^{\sfrac{1}{2}}Sc^{\sfrac{1}{3}},
    \\
    Re_p &= \frac{\rho_g\, d_p\, U}{\mu_g},
    \\
    Sc &= \frac{\mu_g}{\rho_g\,D_m}
\end{align}
where 
\begin{itemize}
    \item $Sh$ is Sherwood number. 
    \item $Re$ is Reynolds number. 
    \item $Sc$ is Schmidt number. 
    \item $\rho_g \; [\si{kg/m^3}]$ is gas density. 
    \item $\mu_g \; \si{[kg/(m\cdot s)]}$ is gas viscosity. 
    \item $U\;[\si{m/s}]$ is free stream velocity or relative velocity between particles and the gas. 
    If the solids circulation rates are fairly low, it equals to particle terminal velocity \citep{jiang1991baffle}. 
    \item $D_m\;[\si{m^2/s}]$ is molecular diffusion coefficient of ozone.  
\end{itemize}

As for effectiveness factor, $\eta$, 
\begin{align}
    \eta &= \frac{3}{\phi^2}(\phi \coth{\phi}-1),
    \label{eq: eta for first order reaction} \\
    \eta &= \frac{1}{\phi}\left(
        \frac{1}{\tanh{3\phi}} - \frac{1}{3 \phi}    
    \right)
    \label{eq: eta for irreversible first order reaction}
\end{align}
and
\begin{equation*}
    \phi = \frac{d_p}{6}\sqrt{\frac{k_v}{D_e}}
\end{equation*}
where $D_e\,[\si{m^2/s}]$ is effective diffusion coefficient of ozone inside the particle \citep[pg. 721]{fogler2016element}. 
For a first order reaction with spherical catalyst particle. 
The effectiveness factor can be calculated by Eq.~\eqref{eq: eta for irreversible first order reaction} that reported by \citet{li2017reaction} and \citet{jiang1991baffle}, or by Eq.~\eqref{eq: eta for first order reaction} that reported by \citet{fogler2016element}.

\section{Models for Fluidized Bed and Contact Efficiency Factor}
Comparing with the fixed bed, the surface area of the catalyst is lower in the fluidized bed due to the aggregation phenomenon \citep{wang2015performance}. 
Thus, a new term called contact efficiency factor is introduced. 

\begin{definition}
    $\alpha\; [-]$ is the contact efficiency factor, which is a fraction of the external surface area of catalysts available for the diffused ozone reactant from the gas phase:
    \begin{align*}
        \frac{\text{total particle surface area}}{\text{particle surface area available to contact with ozone}}
    \end{align*}
    \qed
\end{definition}
For the heterogeneous models, Eq.~\eqref{eq: external dif. - fixed bed} can be written as 
\begin{align*}
    \text{external mass transfer term} =
    D_m\,\frac{(C_g-C_s)}{\delta}
    \quad \times
    \underbrace{\alpha\, A_s}_{\substack{\text{available surface}\\ \text{area of particles}}} 
    \times \quad
    \frac{V_s}{V_s}\,\frac{V_t}{V_t}
\end{align*}
Thus, 
\begin{align}
    \text{CSTR:} \quad
    & \frac{C_{g,o}}{C_{g,i}} = 1- X_\text{CSTR}
    = \frac{1}{1+ \alpha\,\left(
        \dfrac{k_v\,\eta\,k_s\,a_s\,\varepsilon_s\,(1-\varepsilon_s)\,H}{U_g(k_s\, a_s + k_v\,\eta)}
        \right)},
    \label{eq: CSTR in CFB}\\
    \text{PFR:} \quad
    & U_g\,\frac{dC_{g}}{dz} + \frac{k_s\,a_s\,k_v\,\eta}{k_s\,a_s + k_v\,\eta}\,\varepsilon_s\,(1-\varepsilon_s)\,C_g = 0, \\
    \Longrightarrow \quad 
    & \frac{C_{g,o}}{C_{g,i}} = 1- X_\text{PFR}
    = \exp \left[
        - \alpha\,\dfrac{k_v\,\eta\,k_s\,a_s\,\varepsilon_s\,(1-\varepsilon_s)\,H}{U_g(k_s\, a_s + k_v\,\eta)}  
    \right].
    \label{eq: PFR in CFB}
\end{align}

Slightly changing the definition for the $\alpha$, it can be used in the homogeneous models,
\begin{align}
    \text{CSTR: }\quad
    \frac{C_o}{C_i} &= 1-X_\text{CSTR} 
    = \frac{1}{1+\alpha\,\left(\dfrac{k_v\, \varepsilon_s \,(1-\varepsilon_s) \, H}{U_g}\right)},
    \label{eq: homo - CSTR}
    \\
    \text{PFR: }\quad
    \frac{C_o}{C_i} &= 1-X_\text{PFR} 
    = \exp{\left( -\alpha\,\frac{k_v\, \varepsilon_s \, (1-\varepsilon_s) \, H}{U_g} \right)}.
    \label{eq: homo - PFR}
\end{align}


\section{Comparation and Damköhler number}
\begin{definition}
    Damköhler number

    \citet{jiang1991baffle, liujs2016}: 
    \begin{equation}
        Da = \frac{k_v\,\varepsilon_s\, H}{U_g}, 
    \end{equation}

    \citet{li2011catalytic, wang2015performance}: 
    \begin{equation}
        Da = \frac{k_v\,\varepsilon_s\,(1-\varepsilon_s) \, H}{U_g}
    \end{equation}
    where $k_v$ is the reaction constant based on volume of catalyst particles; 
    in the study of Li, Wang, and Liu, $k_v$ is also called apparent reaction constant.
    \qed
\end{definition}

Accordingly, Eq.~\eqref{eq: CSTR in CFB} and \eqref{eq: PFR in CFB} can be written as
\begin{align}
    1- X_\text{CSTR}
    &= \frac{1}{1+\alpha\,Da\,\left(
        \dfrac{k_s\,a_s\,\eta}{k_s\,a_s + k_v\,\eta}
        \right) },
    \\
    1- X_\text{PFR}
    &= \exp \left[
        -\alpha\, Da\,\left(
            \dfrac{k_s\,a_s\,\eta}{k_s\,a_s + k_v\,\eta}
            \right)
    \right].
\end{align}
If we define $k_r = k_s\,a_s$, Eq.~\eqref{eq: homo - CSTR} and \eqref{eq: homo - PFR} could be written as
\begin{align}
    1- X_\text{CSTR}
    &= \frac{1}{1+\alpha\,Da},
    \\
    1- X_\text{PFR}
    &= \exp (-\alpha\, Da),
\end{align}
which have the same form in \citet{li2011catalytic} and \citet{wang2015performance}, as well as in \citet{liujs2016}.

Comparing the four equations above, 
\begin{equation}
    \left(
    \dfrac{k_s\,a_s\,\eta}{k_s\,a_s + k_v\,\eta}
    \right)
    \label{eq: effection of i/d diffusion}
\end{equation}
is the only different due to the consideration of internal and external mass transfer. 
This term can be calculated. 
If we define an apparent reaction constant, 
\begin{equation}
    k_r = k_v\,
    \left(
    \dfrac{k_s\,a_s\,\eta}{k_s\,a_s + k_v\,\eta}
    \right). 
\end{equation}
The expressions of homogeneous and heterogeneous will be the same. 

\section{Conclusion}
In the works of Li and Wang, as well as Liu, the internal diffusion and external diffusion are combined with the reaction constant based on the volume of catalyst to define a new apparent reaction constant. 
By dividing Eq.~\eqref{eq: effection of i/d diffusion}, the real reaction constant could be calculated.


% --------------------------------------------------------------------
%
% --------------------------------------------------------------------

% -------------------------------
\part{FLUID MECHANISM}
\chapter{Vector Calculus}
This chapter records the notes for vector calculus from a book by \citet{matthews1998vector}.

\section{Vector Algebra}
\subsection{Dot product}
The dot product or scalar product of two vectors is a scalar quantity.
\begin{equation}
    \bm{a} \cdot \bm{b} = |\bm{a}| |\bm{b}| \cos \theta
\end{equation}
\begin{itemize}
    \item $ \bm{a}\cdot \bm{b} = \bm{b}\cdot \bm{a} $.
    \item The quantity $|\bm{b}|\cos \theta$ represents the component of the vector $\bm{b}$ in the direction of the vector $\bm{a}$.
    \item $ \bm{a}\cdot \bm{b} = a_1 b_1 + a_2 b_2 + a_3 b_3 $.
    \item $\bm{e}_1 \cdot \bm{e}_2 = 0$. 
\end{itemize}
    
\subsection{Cross product}
The cross product or vector product of two vectors is a vector quantity. 
The magnitude is $|\bm{a}| |\bm{b}| \sin \theta$, and the direction is perpendicular to them in a right-handed sense. 
Then,
\begin{equation}
    \bm{a} \times \bm{b} 
    = |\bm{a}| |\bm{b}| \sin \theta\,\bm{u}
    =   
        \begin{vmatrix}
            \bm{e}_1 & \bm{e}_2 & \bm{e}_3 \\
            \bm{a}_1 & \bm{a}_2 & \bm{a}_3 \\
            \bm{b}_1 & \bm{b}_2 & \bm{b}_3 
        \end{vmatrix}
\end{equation}
where $\bm{u}$ is a unit vector perpendicular to them in a right-handed sense. 
\begin{itemize}
    \item $\bm{a} \times \bm{b} = - \bm{b} \times \bm{a}$.
    \item $\bm{e}_1 \times \bm{e}_2 = 1$. 
\end{itemize}

\subsection{Scalar triple product}
The scalar triple product is defined to be 
\begin{equation}
    \bm{a} \cdot \bm{b} \times \bm{c}
    =
    \bm{a} \cdot (\bm{b} \times \bm{c})
    =
        \begin{vmatrix}
            \bm{a}_1 & \bm{a}_2 & \bm{a}_3 \\
            \bm{b}_1 & \bm{b}_2 & \bm{b}_3 \\
            \bm{c}_1 & \bm{c}_2 & \bm{c}_3 
        \end{vmatrix}
    ,
\end{equation}
written $[\bm{a},\;\bm{b},\;\bm{c}]$. The brackets are unnecessary.

\begin{itemize}
    \item If any two of the vectors are equal, the scalar triple product is zero.
    \item $ \bm{a} \cdot \bm{b} \times \bm{c}
            = \bm{a} \times \bm{b} \cdot \bm{c} $.
    \item $ \bm{a} \cdot \bm{b} \times \bm{c}
            = \bm{b} \cdot \bm{c} \times \bm{a} 
            = \bm{c} \cdot \bm{a} \times \bm{b} $.
\end{itemize}

\subsection{Vector triple product}
The vector triple product is 
$\bm{a} \times (\bm{b}\times \bm{c})$. 
The brackets are important. 
It can expanded as
\begin{equation}
    \bm{a} \times (\bm{b}\times \bm{c})
    = (\bm{a} \cdot \bm{c}) \bm{b}
    - (\bm{a} \cdot \bm{b}) \bm{c}.
\end{equation}
More information see \citealp[pg.~16]{matthews1998vector}.

\section{Line, Surface, and Volume Integrals}
\subsection{Line integrals}
\begin{example}
    A particle moves along a curve path $C$ by force $\bm{F}(\bm{r})$. 
    $\bm{r}$ is the position vector, $\bm{r} = (x,\;y,\;z)$. 
    What is the total amount of work?
    \begin{equation}
        \lim_{N\to \infty} \sum_{i=1}^N \bm{F}_i \cdot \bm{dr}_i
        = \int_C \bm{F} \cdot \bm{dr}.
    \end{equation}
    \qed
\end{example}
\paragraph{Evaluation.}
Line integrals are evaluated by using a parameter, time ($t$), together with a formula giving the value of the position vector $\bm{r}$ in terms of $t$.
\begin{equation}
    \int_C \bm{F} \cdot \bm{dr}
    = \int \bm{F} \cdot \frac{\bm{dr}}{dt} dt 
\end{equation}
where 
\begin{equation}
    \frac{\bm{dr}}{dt} 
    = \left( \frac{dx}{dt},\; \frac{dy}{dt},\; \frac{dz}{dt} \right).
\end{equation}

Line integrals sometimes occur over curves that are close. 
In this case, the integral is written using the symbol $\oint$. 

\paragraph{Conservative vector fields. }
\begin{definition}
    Conservative vector fields. 
    
    A vector field $\bm{F}$ is said to be conservative if it has the property that the line integral of $\bm{F}$ around any closed curve $C$ is zero:
    \begin{equation}
        \oint_C \bm{F} \cdot \bm{dr} = 0.
    \end{equation}
    \qed
\end{definition}
The line integral of $\bm{F}$ along a curve only depends on the endpoints of the curve, not on the path taken by the curve.

\paragraph{Other forms of line integrals. }
\begin{equation}
    \int_C \phi \bm{dr}
    \quad \mathrm{and} \quad
    \int_C \bm{F} \times \bm{dr}. 
\end{equation}

\subsection{Surface integrals}
\begin{example}
    Suppose that fluid flows with velocity $\bm{u}(\bm{r},t)$ through a pipe with $A$ cross-sectional area. 
    What is the total volume of fluid passing through the pipe per unit time?
    
    Suppose the first case: the velocity is directed parallel to the walls of the pipe, with speed $|\bm{u}| = U_0$. 
    In this case, the fluid moves along the pipe as if it were a solid block. 
    In a time $t$, the fluid moves a distance $U_0 t$, so the volume is $U_0 t A$. 
    Thus, the flow rate, or flux is 
    \begin{equation}
        Q = \frac{U_0 t A}{t} = U_0 A.  
    \end{equation}

    Suppose the second case: the velocity is again directed parallel to the walls, but the speed depends on the position within the pipe: 
    $|\bm{u}| = U_0 (x,y)$.
    Besides, the pipe has a square cross-section. 
    Then, the surface element $dS = dx \, dy$. 
    Thus, 
    \begin{align}
        dQ &= U_0(x,y) \, dS = U_0(x,y) \, dx\,dy, \\
        \Longrightarrow \quad
        Q &= \iint_S U_0(x,y) \, dx\,dy. 
    \end{align}

    In the case 1 and 2, the fluid flow direction is perpendicular to the surface. 
    The third case where the vector field $\bm{u}$ and the surface $S$ are both arbitrary. 
    Therefore, only the component of $\bm{u}$ perpendicular to $dS$ contributes to the flux across $dS$.
    Here, it is necessary to introduce a normal vector $\bm{n}$ to the surface $dS$.
    \textbf{The component of $\bm{u}$ perpendicular to $dS$ is then the component of $\bm{u}$ in the direction of $\bm{n}$, which is just $\bm{u} \cdot \bm{n}$.} 
    Therefore, 
    \begin{align}
        dQ &= \bm{u} \cdot \bm{n} \, dS, \\
        \Longrightarrow \quad
        Q &= \iint_S \bm{u} \cdot \bm{n} \, dS. 
    \end{align}

    If the surface is closed, the integral could be written as 
    \begin{equation}
        \oiint_S \bm{u} \cdot \bm{n} \, dS.
    \end{equation}

    \qed 
\end{example}

\paragraph{Evaluation. }
For the second case, if  the square surface given by 
$0 \leq x \leq 1, \; 0 \leq y \leq 1$.
\begin{equation}
    Q = \iint_S U_0(x,y) \, dx\,dy 
      = \int^1_0 \int^1_0 U_0(x,y) \, dx\,dy.
\end{equation}

For the third case, the surface $S$ is curved and can be written in terms of two parameters, $v$ and $w$, so the position vector $\bm{r} = \bm{r}(v,w)$. 
To evaluate the surface integral we need an expression for $\bm{n} \, dS$. 
The cross product of two vectors gives a vector perpendicular to both an with a magnitude equal to the area of the parallelogram create by two vectors. 
Thus, 
\begin{equation}
    \iint_S \bm{u} \cdot \bm{n} \, dS
    = \iint_S \bm{u} \cdot 
    \frac{\partial \bm{r}}{\partial v} \times \frac{\partial \bm{r}}{\partial w} \, dv \, dw.
\end{equation}

\subsection{Volume integrals}
\begin{example}
    Suppose that an object of volume $V$ has a density $\rho$, and the density is a function of position, $\rho = \rho (\bm{r})$.
    What is the total mass ($M$) of the object?
    \begin{equation}
        \iiint_V \rho dV 
        = \lim_{N\to \infty} \sum^N_{i=1} \rho(\bm{r}_i)\,\delta V_i.
    \end{equation}

    Volume integrals can also be used to compute the volumes of objects, in which case $\rho =1$.
    \begin{equation}
        \iiint_V \bm{u} \, dV
    \end{equation}
    where $\bm{u}$ is a vector field.
\end{example}

\section{Partial Differentiation and Taylor Series}
\subsection{Taylor series in more than one variable}

% -------------------------------
% \part{TOOLS}
% \chapter{Liao, Xuefeng's tutorial for Git}
\url{https://www.liaoxuefeng.com/wiki/896043488029600}
\section{Part I - Local Operations}
Create a null directory, and initialized it.
\begin{lstlisting}[basicstyle=\small\ttfamily]
    $ mkdir learngit
    $ cd learngit
    $ pwd
    $ git init
\end{lstlisting}
\texttt{pwd} is to show the current directory.

When edited a document in the directory, 
\begin{lstlisting}[basicstyle=\small\ttfamily]
    $ git add file.xxx
    $ git commit "xxxxxx"
    $ git status
\end{lstlisting}
the first line is to add this document to the Git repository. 
And the second one is to submit the instruction of this or these document(s). The third line is to check the status of this directory.

\begin{lstlisting}[basicstyle=\small\ttfamily]
    $ git reset --hard HEAD^
    $ git reset --hard <commit number>
    $ git reflog
\end{lstlisting}
The first line is to back the last version and the second line is to go to any version if enter the commit number. 
The third line is to show the every operations, which can find the commit number.

The folder saving the files is called ``working directory'', in this tutorial called ``learngit''. 
As for the ``.git'', we called it ``repository'' in Git. 
In the repository, there are stage and HEAD (master) two areas. 
The first on is like a cache region. When we add the files, they would appear in the stage. 
When we commit them, they would appear in the HEAD (master).

The code \texttt{git add <file>} is to put the file in the cache area, and the code \texttt{git commit -m "xxxx"} is to put the file in the repository.

\begin{lstlisting}[basicstyle=\small\ttfamily]
    $ git diff HEAD -- <file.xxx>
\end{lstlisting}
is to show the differences between the file in the working area and the repository.

\begin{lstlisting}[basicstyle=\small\ttfamily]
    $ git checkout -- <file.xxx>
\end{lstlisting}
is to discard all the changes in the working area. 
If we have added the file, we can use the code
\begin{lstlisting}[basicstyle=\small\ttfamily]
    $ git reset HEAD <file.xxx>
\end{lstlisting}
to discard the changes before commit and back to the working area.

\begin{lstlisting}[basicstyle=\small\ttfamily]
    $ rm <file.xxx>
    $ git rm <file.xxx>
    $ git commit ...
\end{lstlisting}
is to remove a file and remove from the cache area. 
When we remove the file from the folder, we can use \texttt{\$ git checkout -- <file.xxx>} to recover the file.

\section{Part II - Remote Repository}
Use the code,
\begin{lstlisting}[basicstyle=\small\ttfamily]
    $ ssh-keygen -t rsa -C "youremail@example.com"
\end{lstlisting}
to create SSH Key, and set them in the Github.com.

After we creating a new repository on the Github.com, we connect the local repository to that one one the Github.com by 
\begin{lstlisting}[basicstyle=\small\ttfamily]
    $ git remote add origin git@github.com:xxx/xxx.git
\end{lstlisting}
where the \texttt{origin} is the name for the remote repository.
Then, push the local files to the Github.com at the first time by 
\begin{lstlisting}[basicstyle=\small\ttfamily]
    $ git push -u origin master
\end{lstlisting}

When we create a new repository at Github.com, we use
\begin{lstlisting}[basicstyle=\small\ttfamily]
    $ git clone git@github.com:DengZhengyuan/gitskills.git
\end{lstlisting}
to clone the remote files to local.

\section{Part III - Branch Management}
There are codes,
\begin{lstlisting}[basicstyle=\small\ttfamily, language=bash]
    $ git branch <name>
    $ git branch
    $ git checkout -b <name> 
    $ git checkout <name> 
    $ git merge <name> 
    $ git branch -d <name>
\end{lstlisting}
means
\begin{enumerate}
    \item create a new branch; 
    \item check the branches; 
    \item create and switch to a new branch; 
    \item switch to the branch; 
    \item merge branch <name> to the current branch; 
    \item delete the branch. 
\end{enumerate}
% \chapter{Zhang, Hongliang's tutorial for UDF}
\url{https://www.youtube.com/playlist?list=PLELxZQTxWBkxhQGkM7SDSaLEjKMFIGpHt}
\section{Lesson 1}
In the UDF book, the chapter 2 introduces the kinds of Macros, which are about 90 to 100 DEFINE Macros. 
The chapter 3 discusses the rules to write the Macros. 
The parallelization is introduced in the chapter 7. 
These three chapters are the most important.

\subsection{Compilers}
Common I/O functions,
\begin{lstlisting}[basicstyle=\small\ttfamily, numbers=left, language=C]
    scanf (...)
    printf (...)  /*** print to the cmd in the fluent ***/

    fscanf (...)  /*** file scan ***/
    fprintf (...)  /*** file print ***/
\end{lstlisting}

% -------------------------------
% \part{NOT USE}
% \chapter{Thesises}
\section{Experiments}
%
%
%----------------------------------------------------------------//
\subsection[Li, Dongbing, 2010]{Li, Dongbing, 2010, \cite{li2010}}
\textit{Investigation of circulating fluidized bed riser and downer reactor performancefor catalytic ozone decomposition.}

Dimension: rise (76 mm i.d., 10.2 m high), downer (76 mm i.d., 5.8 m high). Superficial ags velocity $(U_g)$: $2-5\,\si{m/s}$. 
Solids circulation rate $(G_s)$: $50-100\,\si{kg/(m^2\,s)}$.

In this thesis, the hydrodynamics, ozone decomposition in the riser and downer were studied; 
the performances were compared between riser and downer; 
the ozone distributions in different angular position in riser were studied.

%
%
%----------------------------------------------------------------//
\subsection[Wang, Chengxiu, 2013]{Wang, Chengxiu, 2013, \cite{wangcx2013}}
\textit{High density gas-solids circulating fluidized bed riser and downer reactors.}

Dimension is the same as the Li, Dongbing's: rise (76 mm i.d., 10.2 m high), downer (76 mm i.d., 5.8 m high). 
Superficial ags velocity $(U_g)$: $3-9\,\si{m/s}$. 
Solids circulation rate $(G_s)$: $100-1000\,\si{kg/(m^2\,s)}$. 
The holdup was much larger than Li's.

Similarly, in this thesis, the hydrodynamics and ozone decomposition in the riser and downer were studied;
the performances were compared between riser and downer; 
the ozone distributions in different angular position in riser were studied.

\section{Simulations}
%
%
%----------------------------------------------------------------//
\subsection[Kong, Lei, 2012]{Kong, Lei, 2012, \cite{konglei2012}}
\textit{Numerical simulation of catalytic ozone decomposition reaction in a gas-solids circulating fluidized bed riser.}

Based on the Li's experiments, the numerical simulations were studied in this thesis. 
The two-fluid model (granular kinetic theory) was adopted in the research. 
The effect of the wall boundary conditions were studies. 
The radial and axial profiles of the ozone composition were simulated and compared with the experimental results in Li's.

%
%
%----------------------------------------------------------------//
\subsection[Liu, Yunfeng, 2018]{Liu, Yunfeng, 2018, \cite{liuyf2018}}
\textit{Numerical simulation of three-phase flows in the inverse fluidized bed.}


% \chapter{Articles}
\section{Review}
%
%
%----------------------------------------------------------------//
\subsection[Ge, Wei, 2019]{Ge, Wei, 2019, \cite{ge2019multiscale}}
\textit{Multiscale structures in particle--fluid systems: Characterization, modeling, and simulation.}

This is a very general paper that focused on the mesoscale structure in the particle-fluid system, or the clusters in the fluidized bed. 
The paper discussed not only the experimental researches but also the modeling studies. 
%
%
%----------------------------------------------------------------//
\subsection[Zhong, Wenqi, 2016]{Zhong, Wenqi, 2016, \cite{zhong2016cfd}}
\textit{CFD simulation of dense particulate reaction system: Approaches, recent advances and applications.}

This paper focused on the modeling works in the dense particulate reaction systems. 
Recent advances and applications of the Eulerian-Eulerian approach and the Eulerian-Lagrangian approaches, as well as the limitations.

\section{Modeling}
%
%
%----------------------------------------------------------------//
\subsection[Gidaspow, kinetic theory for bubbling, 1990]{Gidaspow, kinetic theory for bubbling, 1990, \cite{Ding1990}}
\textit{A bubbling fluidization model using kinetic theory of granular flow.}

This is the first time to use the kinetic theory of granular flow to simulate the fluidized bed. 
In this paper, a two dimensional bubbling bed was investigated. 

%
%
%----------------------------------------------------------------//
\subsection[Gidaspow, kinetic theory for riser, 2003]{Gidaspow, kinetic theory for riser, 2003, \cite{Huilin2003}}
\textit{Hydrodynamic simulation of gas-solid flow in a riser using kinetic theory of granular flow.}

\section{Grid Study}
%
%
%----------------------------------------------------------------//
\subsection[Grid study, 2014]{Grid study, \cite{Li2014}}
\textit{Reprint of ``CFD simulations of circulating fluidized bed risers, part I: Grid study''.}

%
%
%----------------------------------------------------------------//
\subsection[Inlet profile study, 2010]{Inlet profile study, \cite{Peng2010}}
\textit{Numerical Study on the Effect of the Air Jets at the Inlet Distributor in the Gas-Solids Circulating Fluidized-Bed Risers.}

%
%
%----------------------------------------------------------------//
\subsection[Grid study, 2005]{Grid study, \cite{AndrewsIV2005}}
\textit{Coarse-grid simulation of gas-particle flows in vertical risers.}

%
%
%----------------------------------------------------------------//
\subsection[Grid study, 2001]{Grid study, \cite{Agrawal2001}}
\textit{The role of meso-scale structures in rapid gas–solid flows.}

%
%
%----------------------------------------------------------------//
\subsection[Grid study, 2001]{Grid study, \cite{Zhang2001}}
\textit{High-resolution three-dimensional numerical simulation of a circulating fluidized bed.}



\section{Ozone Decomposition}
%
%
%----------------------------------------------------------------//
\subsection[2003]{\cite{Therdthianwong2003}}
\textit{Modeling and simulation of circulating fluidized bed reactor with catalytic ozone decomposition reaction.}

%
%
%----------------------------------------------------------------//
\subsection[2004]{\cite{Hansen2004}}
\textit{A three-dimensional simulation of gas/particle flow and ozone decomposition in the riser of a circulating fluidized bed.}

%
%
%----------------------------------------------------------------//
\subsection[2008]{\cite{Dong2008}}
\textit{A multiscale mass transfer model for gas-solid riser flows: Part II-Sub-grid simulation of ozone decomposition.} 




% \chapter{Books}
\section{Modeling}
%
%
%----------------------------------------------------------------//
\subsection[Computational methods for multiphase flow]{Computational methods for multiphase flow, \cite{prosperetti2009computational}}
\textit{Computational methods for multiphase flow.}

This book needs the graduate-level fluid mechanics and numerical methods knowledge as the background. 
The first part introduced the methods for solving the N-S equations by finite difference and finite element methods, which includes computational approach, grid methods for particles, and finite element methods for particulate flows.
The second part introduced the particles models and two-fluid models, which includes point-particle methods, two-fluid methods, and coupled methods. 
Also, the Lattice Boltzmann model was included. 

\section{Fluid Mechanics}
%
%
%----------------------------------------------------------------//
\subsection[A first course in continuum mechanics]{A first course in continuum mechanics, \cite{gonzalez2008first}}
\textit{A first course in continuum mechanics.}

This book is a background material for mathematics. 
The first part introduces the tensor algebra and calculus in three-dimensional Euclidean space. 
For this, \cite{matthews2012vector} can also support. 
The second part introduces the basic equations refer to the fluid mechanics, which includes mass, motion, momentum, energy, and entropy. The third part is not necessary.

The calculus and differential equations should be considered as the fundamental materials.

After this, \cite{wang2014fluid} and \cite{kundu2012fluid} can also support the knowledge for fluid mechanics.

\section{Manual}
%
%
%----------------------------------------------------------------//
\subsection[ANSYS Fluent UDF Manual]{ANSYS Fluent UDF Manual, \cite{udfmanual}}
\textit{ANSYS Fluent UDF Manual.}

The manual for user-defined functions in the ANSYS Fluent. 
UDF is necessary for modifying the inlet profiles and coupling chemical reactions with the fluid models in the software.
%
%
%----------------------------------------------------------------//
\subsection[ANSYS Fluent Theory Guide]{ANSYS Fluent Theory Guide, \cite{FluentTheoryGuide}}
\textit{ANSYS Fluent Theory Guide.}

This manual talks about the theoretical information about the models used in the ANSYS Fluent.

%
%
%----------------------------------------------------------------//
\subsection[C Primer Plus]{C Primer Plus, \cite{Prata:2013:CPP:2578965}}
\textit{C Primer Plus.}

A complete tutorial on C programming.



%-----------------------------------------------
\end{addmargin}
%-----------------------------------------------
%	BIBLIOGRAPHY
%-----------------------------------------------
\bibliographystyle{apalike}
\bibliography{bibliograph}
%-----------------------------------------------
\end{document}